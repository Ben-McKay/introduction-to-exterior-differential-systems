\chapter{Riemannian geometry upstairs}\label{chapter:upstairs}

\section{Where we are going}
In this appendix, we assume that the reader has previous experience in Riemannian geometry, from any of the standard textbooks, for example \cite{doCarmo:1992,Gallot.Hulin.Lafontaine:2004,Milnor:1963,Morgan:1998,ONeill:1983}.
We imitate the theory of chapter~\vref{chapter:moving.frame} as closely as possible, making any Riemannian manifold seem roughly like a submanifold of Euclidean space.
For the next few pages, the reader will have to grit his or her teeth and bear with some formal but not very geometric material.

\section{Surfaces}
To get started, we suppose that \(g\) is a Riemannian metric on a surface \(M\).
For simplicity of notation, write \(g\)-inner products as dot products.
An \emph{orthonormal frame}\define{frame!orthonormal}\define{orthonormal frame}  \((m,e)\) is a choice of point \(m\) of \(M\) and a choice of orthonormal basis \(e_1, e_2\) of the tangent plane \(T_m M\).
The \emph{tangent frame bundle} of \(M\) is the collection of orthonormal frames of tangent spaces of \(M\).
Note that for a surface \(M\) in \(\E[3]\), this is \emph{not} the same as the frame bundle we constructed previously in appendix~\vref{chapter:moving.frame}, which we could refer to as the \emph{bundle of ambient adapted frames}.
The tangent frame bundle of a surface in \(\E[3]\) is the quotient of the ambient adapted frame bundle by ignoring \(e_3\), giving a \(2-1\) covering map.
Nonetheless, we will use the same notation \(\frameBundle{M}\) for the tangent frame bundle.

Take any two linearly independent vector fields  on an open subset \(U\subseteq M\).
Apply the Gram--Schmidt orthogonalization procedure to them to construct orthonormal vector fields, say \(\eu{1}, \eu{2}\), defined on \(U\).
Over \(U\), any orthonormal frame \(e_1,e_2\) is uniquely expressed as
\begin{align*}
e_1&=\cos \theta \eu{1} \pm \sin \theta \eu{2},\\
e_2&=-\sin \theta \eu{1} \pm \cos \theta \eu{2}.
\end{align*}
In this way, we identify \(\frameBundle{U}\cong U \times (S^1 \cup S^1)\), so the tangent frame bundle becomes a manifold.
The map \(p \colon (m,e) \in \frameBundle{M} \mapsto m \in M\) is the \emph{foot map}, a smooth map.
The \emph{soldering forms} \(\omega_1,\omega_2\) on \(\frameBundle{M}\) are defined by
\[
p'(m,e_1,e_2)v=\omega_1(v)e_1+\omega_2(v)e_2.
\]
If the dual \(1\)-forms to \(\eu{1},\eu{2}\) on \(U\) are \(\omu{1},\omu{2}\) then
\[
\begin{pmatrix}
\omega_1\\
\omega_2
\end{pmatrix}
=
\begin{pmatrix}
\cos \theta  & \pm \sin \theta \\
-\sin \theta & \pm \cos \theta
\end{pmatrix}
\begin{pmatrix}
\omu{1}\\
\omu{2}
\end{pmatrix}.
\]

Suppose henceforth that the surface is oriented.
We pick out only the positively oriented frames, to get rid of \(\pm\) so (using the same notation to also denote the bundle of positively oriented orthonormal frames) \(\frameBundle{U}\cong U\times S^1\).
Let \(\omega\defeq\omega_1+i\omega_2\), and write this as
\[
\omega=e^{i\theta}\omu{}.
\]
The area form \(dA\defeq \omu{1}\wedge\omu{2}\) pulls back to \(dA=\omega_1\wedge\omega_2\), independent of the choice of \(\eu{1},\eu{2}\), so a globally defined invariant on \(M\).
Since \(dA\) is a nonzero area form, \(d\omu{1},d\omu{2}\) must be multiples of it, say
\begin{align*}
d\omu{1}&=-a_1dA,\\
d\omu{2}&=-a_2dA.
\end{align*}
Let \(\amu\defeq a_1\omu{1}+a_2\omu{2}\).
Take exterior derivative of both sides to see that \(d\omu{}=i\amu\wedge\omu{}\) on \(U\), and that
\[
d\omega=i \, (\amu+d\theta) \wedge \omega.
\]
The real-valued \(1\)-form  \(\amu\) on \(U\) is uniquely determined by \(0=d\omu{}-i\amu\wedge\omu{}\).
So the real-valued \(1\)-form \(\alpha\defeq\amu+d\theta\) is as well, giving \(0=d\omega-i\alpha\wedge\omega\) for a unique real-valued \emph{Levi-Civita connection \(1\)-form} \(\alpha\).
Take exterior derivative of that equation to see that
\[
d\alpha=K\omega_1\wedge\omega_2,
\]
for a unique function \(K\) on the tangent frame bundle.
But on \(\frameBundle{U}\), \(d\alpha=d(\amu+d\theta)=d\amu\) is the same for any choice of framing, i.e. \(K\) is a function well defined on the surface \(M\).
\begin{problem}{upstairs:unorien}
Suppose that \(M\) is not orientable.
Prove that \(K\) is nonetheless well defined, while the area form \(dA=\omega_1\wedge\omega_2\) is not.
Prove that the curvature form \(K \, dA\) is well defined on \(M\) if and only if \(K=0\).
\end{problem}
\begin{theorem}[Theorema egregium]\SubIndex{theorema egregium}
The Levi-Civita connection \(1\)-form and the Gauss curvature of a Riemannian metric on a surface depend only on the Riemannian metric, not on any choice of isometric immersion into Euclidean space.
\end{theorem}

Geodesic normal coordinates work out exactly the same.
But on abstract surfaces, we learn more.
Take any smooth function \(K(r,\theta)\); there is a unique solution \(h\) to \(h_{rr}+Kh=0\) with \(h=0\) and \(h_r=1\) at \(r=0\): every smooth function \(K\) defined near the origin of the plane is the Gauss curvature of some abstract Riemannian metric \(dr^2 + h^2 \, d\theta^2\) near the origin.
We need to be more careful to be sure that this metric is smooth at the origin.
The reader can check this by applying the equation \(h_{rr}+Kh=0\) iteratively to see that a Fourier series of  \(h\) in \(\theta\), with coefficients expanded in a Taylor series in \(r\), turns out to contain only terms which can be written as monomials in \(x,y\).
If \(K<0\) then \(h_{rr}=-Kh>0\) for \(h>0\), so \(h\) grows, and so has no zeroes: if \(K<0\) is defined in the plane, there is a Riemannian metric \(dr^2+h^2 \, d\theta^2\) defined in the plane with Gauss curvature \(K\).
By construction, \(r,\theta\) are geodesic normal coordinates, so this metric is complete.
\begin{theorem}
Any connected surface bearing a complete metric of negative Gauss curvature has exponential map at any point a smooth universal covering map from the plane.
\end{theorem}
\begin{proof}
The exponential map is defined in the plane by completeness, but \(h\) has no zeroes so the exponential map is a local diffeomorphism.
The pullback metric is a complete.
So the exponential map covers each geodesic disk on the surface with geodesic disks in the plane, and so is a covering map.
\end{proof}

The smoothness of isometries follows, with the same proof.
The Gauss--Bonnet theorem also follows, with the same proof,\SubIndex{Gauss--Bonnet theorem}\SubIndex{theorem!Gauss--Bonnet} for any oriented surface with Riemannian metric, even with corners.
We have only uncovered the structure equations for an oriented surface.
If we allow an unoriented surface, this gets more complicated, but instead let us consider the general Riemannian manifold in any dimension.

\section{Repeating in any dimension}
Suppose that \(g\) is a Riemannian metric on a manifold \(M\).
An \emph{orthonormal frame}\define{frame!orthonormal}\define{orthonormal frame}  \((m,e)\) is a choice of point \(m\) of \(M\) and a choice of orthonormal basis \(e_1, e_2, \dots, e_n\) of the tangent space\SubIndex{tangent!space} \(T_m M\).
If you pick an orthonormal frame \((m,e)\) and I pick another one, say \((m,e')\) at the same point \(m\), then yours and mine must agree up to a unique orthogonal matrix, because they are orthonormal bases of the same vector space \(T_m M\).
So \(e_j' = \sum_i h_{ij} e_i\) for a uniquely determined orthogonal \(n \times n\) matrix \(h=\pr{h_{ij}}\); denote this \(e'\) as \(eh\).

Any rigid motion \(\phi\) of \(M\) (i.e. diffeomorphism preserving the Riemannian metric) takes orthonormal frames to orthonormal frames, taking \(e_1, e_2, \dots, e_n\) to \(\phi_* e_1, \phi_* e_2, \dots, \phi_* e_n\).

The \emph{frame bundle}\define{frame bundle}\define{bundle!frame} of a Riemannian manifold \(M\) is the set \(\frameBundle{M}\) of orthonormal frames.
To be more concrete, take any local coordinates \(x_1, x_2, \dots, x_n\) on \(M\) and take some linearly independent vector fields \(\bar{e}_1, \bar{e}_2, \dots, \bar{e}_n\).
Applying Gram--Schmidt orthogonalization to these, we can assume they form a local orthonormal framing.
Every point \((m,e)\) of \(\frameBundle{M}\) has some orthogonal matrix \(h=h_{ij}\) so that \(e_j=\sum_i h_{ij} \bar{e}_i\).
Make \(\frameBundle{M}\) into a manifold, with local coordinates \(x_1, x_2, \dots, x_n, h_{12}, h_{13}, \dots, h_{n-1,n}\).

The map \(p \colon (m,e) \in \frameBundle{M} \mapsto m \in M\) is the \emph{foot map}.\define{foot map}
\begin{problem}{riemannian.geometry:frame.bundle.manifold}
Prove that the frame bundle \(\frameBundle{M}\) is a smooth manifold in such a manner that the foot map is smooth and the map \((m,e) \mapsto (m,eh)\) is smooth for any orthogonal matrix \(h\).
\end{problem}
If \(v\) is a tangent vector on the manifold \(\frameBundle{M}\), say \(v \in T_{(m,e)} \frameBundle{M}\), we can write \(p'(m,e) v\), the infinitesimal motion \(\dot{m}\) of the point \(m\), in the basis \(e_1, e_2, \dots, e_n\), say as
\[
p'(m,e)v = v_1 e_1 + v_2 e_2 + \dots + v_n e_n.
\]
Each of these \(v_1, v_2, \dots, v_n\) are numbers which depend linearly on the choice of vector \(v \in T\frameBundle{M}\); define \(1\)-forms \(\omega_1, \omega_2, \dots, \omega_n\), the \emph{soldering forms}\define{soldering forms}, by \(v \hook \omega_i = v_i = g\of{e_i,p'(m,e)v}\).
So the soldering forms measure, as we move a frame, how the foot point of the frame moves, as measured in the frame itself as a basis.

Take an orthonormal framing \(\eu{1}, \eu{2}, \dots, \eu{n}\) on \(M\), with dual coframing \(\omu{1}, \omu{2}, \dots, \omu{n}\).
Take local coordinates \(x_1, x_2, \dots, x_n\) on \(M\) near a point \(m\).
Every element of \(\frameBundle{M}\) near \((m,e)\) is uniquely represented by the coordinates \(x_1, x_2, \dots, x_n\) of its point of \(M\) and by writing its frame as \(eh\), so that 
\[
x_1, x_2, \dots, x_n, h_{12}, h_{13}, \dots, h_{n-1,n}
\]
are local coordinates on \(\frameBundle{M}\).
Write each tangent vector \(v \in T\frameBundle{M}\) as 
\[
v = v_i \pderiv{}{x_i} + v_{ij} \pderiv{}{h_{ij}}.
\]
We find that
\[
p'(m,e)v=v_i \pderiv{}{x_i},
\]
so that
\[
v \hook \omega_i=v_i \pderiv{}{x_i} \hook h_{ji}\xi_j.
\]
In other words, 
\[
\omega_i = h_{ji} \omu{j}
\]
in these coordinates.

\begin{problem}{riemannian.geometry:section}
A local section \(s\) of the bundle \(\frameBundle{M} \to M\), i.e. a choice point \((x,e(x)) \in \frameBundle{M}\) above each point \(x\) in some open set \(U \subset M\), is precisely an orthonormal framing \(\eu{1},\eu{2}, \dots, \eu{n}\) on \(U\).
Suppose that \(\omu{1}, \omu{2}, \dots, \omu{n}\) is the dual orthonormal coframing.
Prove that \(s^* \omega_i = \omu{i}\).
\end{problem}

For any orthogonal \(n \times n\) matrix \(h\), write \(r_h(m,e)\) to mean \((m,eh)\).
Recall that \(r_h \colon \frameBundle{M} \to \frameBundle{M}\) is a diffeomorphism.
Recall the foot map \(p(m,e)=m\); clearly \(p(m,eh)=m=p(m,e)\), so \(p \circ r_h=p\).
Let
\[
\omega\defeq
\begin{pmatrix}
\omega_1 \\
\omega_2 \\
\vdots \\
\omega_n
\end{pmatrix}.
\]
\begin{problem}{upstairs:transform.h}
Denote by \(\transpose{h}\) the transpose of any \(n \times n\) orthogonal matrix \(h\).
For any Riemannian manifold \(M\) , prove that the soldering forms on the frame bundle of \(M\) satisfy \(r_h^* \omega = \transpose{h} \omega\).
\end{problem}
\begin{answer}{upstairs:transform.h}
At any point \((m,e) \in \frameBundle{M}\), for any vector \(v \in T_{(m,e)} \frameBundle{M}\),
\begin{align*}
v \hook r_h^* \omega_i
&=
\pr{r_h'(m,e) v} \hook \omega_i,
\\
&=
g\of{(eh)_i, p'\of{m,eh} r_h'(m,e) v},
\\
&=
g\of{h_{ji} e_j,\pr{p \circ r_h}'(m,e) v},
\\
&=
h_{ji} g\of{e_j,p'(m,e) v},
\\
&=
h_{ji} \pr{v \hook \omega_j},
\\
&=
v \hook \pr{h_{ji} \omega_j}.
\end{align*}
\end{answer}
We want to work entirely on the frame bundle, so we need to recognise when a map of frame bundles arises from an isometry of Riemannian metrics.
\begin{lemma}
Suppose that \(\phi \colon P \to Q\) is a (local) Riemannian isometry.
Denote by \(\phi_*\) or \(F\) the map \((p,e) \mapsto \pr{q,e'}\) where \(q=\phi(p)\) and \(e'_i = \phi'(p)e_i\).
Then 
\begin{enumerate}
\item
\(F \colon \frameBundle{P} \to \frameBundle{Q}\) is a (local) diffeomorphism and
\item
\(F \circ r_h = r_h \circ F\) for any orthogonal \(n \times n\) matrix \(h\) and
\item
\(F^* \omega=\omega\), i.e. \(F\) preserves the soldering form.
\end{enumerate}

Conversely, if \(F \colon \frameBundle{P} \to \frameBundle{Q}\) is a (local) diffeomorphism and \(F^* \omega=\omega\), i.e. \(F\) preserves the soldering form, then we can cover \(\frameBundle{P}\) by open sets on each of which \(F\) has the form \(F=\phi_*\) for a unique local Riemannian isometry \(\phi\). 
If furthermore \(P\) is connected and, at some point \(\pr{p_0,e}\) of \(\frameBundle{P}\), \(F\of{p_0,eh} = r_h F{p_0,e}\) for one orthogonal \(n \times n\) matrix \(h\) with negative determinant, then \(F=\phi_*\) for a unique (local) Riemannian isometry \(\phi\). 
\end{lemma}
\begin{proof}
Given a (local) isometry \(\phi\), write out our local coordinates on the frame bundles, using orthonormal frames on \(P\) and \(Q\) identified by \(\phi\), and check that \(\phi_*\) is a local diffeomorphism, and preserves \(\omega\).
Unwind the definition of \(\omega\) to check that \(\phi_* r_h= r_h \phi_*\) for any orthogonal matrix \(h\).

Conversely, take a smooth map \(F \colon \frameBundle{P} \to \frameBundle{Q}\) so that
\begin{enumerate}
\item
a (local) diffeomorphism  and
\item
\(F^* \omega=\omega\), i.e. \(F\) preserves the soldering form.
\end{enumerate}
Denote our foot maps as \(p \colon \frameBundle{P} \to P\) and \(q \colon \frameBundle{Q} \to Q\) and our metrics as \(g_P\) and \(g_Q\).
The fibers of the foot map \(\frameBundle{P} \to P\) are precisely the leaves of the foliation \(\omega=0\).
So locally \(F\) takes leaves to leaves, i.e. open subsets of fibers to fibers.
So restricting to a smaller open set in \(\frameBundle{P}\), we can suppose that \(F\) preserves fibers.
The smooth map \(q \circ F \colon \frameBundle{P} \to Q\) is constant on fibers, so (by the rank theorem) descends uniquely to a smooth map \(\phi \colon P \to Q\) (after perhaps restricting to a smaller open set) so that \(\phi \circ p = q \circ F\).

\begin{problem}{upstairs:prove.local.isom}
Prove that \(\phi\) is a local Riemannian isometry.
\end{problem}
\begin{answer}{upstairs:prove.local.isom}
At any point \(\pr{p_0,e} \in \frameBundle{P}\), if we let \(\pr{q_0,e'}=F\of{p_0,e}\),
\begin{align*}
v \hook \omega_i\of{p_0,e}
&=
g_P\of{e_i, p'\of{p_0,e}v},
\\
&=
v \hook F^* \omega_i \of{p_0,e},
\\
&=
\pr{F'\of{p_0,e}v} \hook \omega_i,
\\
&=
g_Q\of{e'_i,q'\of{q_0,e'} \circ F'\of{p_0,e}v},
\\
&=
g_Q\of{e'_i,\pr{q \circ F}'\of{p_0,e}v},
\\
&=
g_Q\of{e'_i,\pr{\phi \circ p}'\of{p_0,e}v},
\\
&=
g_Q\of{e'_i,\phi'\of{p_0} p'\of{p_0,e}v}.
\end{align*}
So finally, for any vector \(w \in T_{p_0} P\), write \(w=p'\of{p_0,e}v\) for some \(v\) (since \(p\) is a submersion), and then we have
\[
g_P\of{e_i,w}=g_Q\of{e_i',\phi'\of{p_0}w}.
\]
If we take \(w=e_j\), we get
\[
g_P\of{e_i,e_j}=g_Q\of{e_i',\phi'\of{p_0}e_j},
\]
so that \(\phi'\of{p_0}e_j = e_j'\).
\end{answer}
The value of \(\phi\) is defined by noting that \(q \circ F\) is locally constant along a fiber of \(p\).
In particular, the value of \(q \circ F\) remains the same along any path inside each fiber of \(p\).
The fiber of \(p\) above a point \(p_0\) of \(P\) consists of the orthonormal frames at the given point \(p_0\), and is thus a copy of the group of \(n \times n\) orthogonal matrices.
The rotation matrices form a path connected component of that group.
Every orthogonal matrix is either a rotation matrix or has negative determinant.
So we have only to check equivariance of \(F\) under one negative determinant matrix to see that \(\phi\) extends to be defined globally.
\end{proof}

\section{The Levi--Civita connection}
Take a coframing \(\omu{1}, \omu{2}, \dots, \omu{n}\) on an open subset of a manifold \(M\).
Since these \(1\)-forms are a basis of the \(1\)-forms, every differential form admits a unique expansion in them.
In particular, the \(2\)-forms \(d\omu{i}\) admit a unique expansion (in Einstein summation notation):
\[
d \omu{i} = -c_{ijk} \omu{j} \wedge \omu{k}
\]
for some functions \(c_{ijk}\) on \(M\); the minus sign is for future convenience.

\begin{lemma}
Consider the vector space \(V\) for which each element \(t\) of \(V\) is a collection \(t=\pr{t_{ijk}}\) of numbers \(t_{ijk}\), defined for \(i,j,k=1,2,\dots,n\).
Let \(B \subset V\) be those antisymmetric in \(i,j\), and let \(C \subset V\) be those antisymmetric in \(j,k\).
Then the linear map \(b \in B \mapsto c \in C\) given by
\[
c_{ijk} = \frac{1}{2}\pr{b_{ijk} - b_{ikj}}
\]
is a linear isomorphism with inverse
\[
b_{ijk} = c_{ijk} + c_{jki} + c_{kji}.
\]
\end{lemma}

\begin{lemma}[The fundamental lemma of Riemannian geometry I]\label{lemma:flrg.one}\define{lemma!fundamental, of Riemannian geometry}\define{fundamental!lemma of Riemannian geometry}
For any smooth orthonormal coframing \(\omu{1}, \omu{2}, \dots, \omu{n}\) on a Riemannian manifold, there is a unique collection of smooth \(1\)-forms \(\gmu{i}{j}=-\gmu{j}{i}\), the Levi--Civita connection \(1\)-forms, so that 
\[
d\omu{i} = -\gmu{i}{j}\wedge\omu{j}.
\]
\end{lemma}
\begin{proof}
Write \(d\omu{i} = -c_{ijk}\omu{j}\wedge\omu{k}\) for unique \(c_{ijk}\) antisymmetric in \(j,k\).
Apply our (constant) linear isomorphism at each point to write \(d\omu{i} = -b_{ijk} \omu{j}\wedge\omu{k}\) for unique \(b_{ijk}\) antisymmetric in \(i,j\).
Let \(\gmu{i}{j}\defeq b_{ijk}\omu{k}\).
Then it is clear that \(\gmu{j}{i}=-\gmu{i}{j}\).
Conversely, expand any such \(\gmu{i}{j}\) into \(\gmu{i}{j}=b_{ijk}\omu{k}\) and apply the linear isomorphism to see that you return back to the same \(c_{ijk}\).
\end{proof}

\begin{lemma}[The fundamental lemma of Riemannian geometry II]%
\label{lemma:FLRG}%
\define{lemma!fundamental, of Riemannian geometry}%
\define{fundamental!lemma of Riemannian geometry}
For any Riemannian manifold \(M\) with orthonormal frame bundle \(\frameBundle{M}\) and soldering forms \(\omega_1, \omega_2, \dots, \omega_n\), there is a unique collection of smooth \(1\)-forms \(\gamma_{ij}=-\gamma_{ji}\) on the frame bundle so that 
\[
d \omega_i = - \gamma_{ij} \wedge \omega_j.
\]
\end{lemma}
\begin{proof}
Take an orthonormal coframing \(\omu{1}, \omu{2},\dots,\omu{n}\) with Levi-Civita \(1\)-forms \(\gmu{i}{j}\) and dual framing \(\eu{1},\eu{2},\dots,\eu{n}\) as above.
Write out each orthonormal frame as \(e_j = h_{ij}\eu{i}\).
Write \(\omu{i}\) to mean both the \(\omu{i}\) coframing and the pullback of this coframing by the foot map.
Recall that in our induced coordinates on \(\frameBundle{M}\), 
\[
\omega_i = h_{ji}\omu{j}.
\]
Take exterior derivative:
\begin{align*}
d \omega_i 
&= 
dh_{ji} \wedge \omu{j} - h_{ji} \gmu{j}{k} \wedge\omu{k},
\\
&=
dh_{ji} h^{-1}_{jm} \wedge h_{m\ell} \omu{\ell} 
-
h_{ji} \gmu{j}{k} h^{-1}_{km} h_{m \ell}\omu{\ell},
\\
&=
-\gamma_{ij}\wedge\omega_j,
\end{align*}
where
\[
\gamma_{ij} = - dh_{ik} h^{-1}_{kj} + h_{ik} \gmu{k}{m} h_{mj}.
\]
If there are two choices for \(\gamma_{ij}\), then the difference, say \(\zeta_{ij}\), satisfies 
\[
0 = \zeta_{ij} \wedge \omega_j.
\]
By Cartan's lemma, 
\[
\zeta_{ij} = b_{ijk} \omega_k,
\]
with \(b_{ijk}\) antisymmetric in \(i,j\) and symmetric in \(j,k\).
By our linear algebra lemma above, this ensures that \(b_{ijk}=0\) for all \(i,j,k\).
\end{proof}

Note that as a consequence, every local Riemannian isometry preserves not only the soldering form but also the Levi-Civita connection \(1\)-forms.

\begin{lemma}
For any Riemannian manifold \(M\) and any \(n \times n\) orthogonal matrix \(h\), the soldering forms and Levi--Civita connection \(1\)-forms on the frame bundle of \(M\) satisfy \(r_h^* \omega = \transpose{h} \omega\) and \(r_h^* \gamma = h^{-1} \gamma h\) for any orthogonal \(n \times n\) matrix \(h\).
\end{lemma}
\begin{proof}
Expand out \(0=r_h^*(d\omega_i+\gamma_{ij} \wedge\omega_j)\), using \(r_h^*\omega_i=h_{ji} \omega_j\).
\end{proof}
\begin{lemma}
The soldering and Levi--Civita connection forms together coframe the frame bundle.
The soldering forms vanish precisely on the fibers of the map \(\frameBundle{M} \to M\), for any Riemannian manifold \(M\).
If \(M\) is \(n\)-dimensional, given any antisymmetric \(n\times n\) matrix \(A\), define a vector field \(\vec{A}\) on \(\frameBundle{M}\) by
\[
\vec{A}=\left.\frac{d}{dt}\right|_{t=0} r_{e^{tA}}.
\]
This is well defined, since \(e^{tA}\) is orthogonal.
Moreover, \(\vec{A}\hook\omega=0\) and \(\vec{A}\hook\gamma=A\).
\end{lemma}
\begin{proof}
As above, take coordinates \((x,h)\) on \(\frameBundle{M}\) arising from coordinates \(x\) on \(M\).
The action of any \(g \in \Orth{n}\) is \(r_g(x,h)=(x,hg)\), so \(\vec{A}\hook (dx,dh)=(0,hA)\).
From the definitions of \(\omega,\gamma\), \(\vec{A} \hook (\omega,\gamma)=(0,A)\).
Clearly \(\omega=0\) on the fibers, and has linearly independent components \(\omega_i\).
So \(\omega,\gamma\) coframe.
\end{proof}


\section{The curvature}
The soldering and connection \(1\)-forms are linearly independent, giving a basis of each tangent space of the frame bundle.
But any two bases of a vector space look the same as any other two, so we can't yet see how one Riemannian manifold differs from another.
We want to say that they are differently ``curved''.
\begin{lemma}\label{lemma:curvature.exists}
For any Riemannian manifold \(M\), there are unique functions \(R_{ijk\ell}\) on the frame bundle of \(M\) so that the soldering forms \(\omega_i\) and Levi-Civita connection forms \(\gamma_{ij}\) of \(M\) satisfy
\begin{align*}
d\omega_i + \gamma_{ij} \wedge \omega_j &= 0, \\
d\gamma_{ij} + \gamma_{ik} \wedge \gamma_{kj} &= \frac{1}{2}R_{ijk\ell} \omega_k \wedge \omega_{\ell}
\end{align*}
and
\begin{align*}
0 &= R_{ijk\ell} + R_{jik\ell},
\\
&= R_{ijk\ell} + R_{ij\ell k},
\\
&= R_{k\ell ij},
\\
&= R_{ijk\ell} + R_{kij\ell} + R_{jki\ell}.
\end{align*}
If we let \(\Omega_{ij} \defeq R_{ijk\ell} \omega_k \wedge \omega_{\ell}\), then
for any orthogonal \(n \times n\) matrix \(h\), 
\begin{align*}
r_h^* \omega &= \transpose{h} \omega, \\
r_h^* \gamma &= h^{-1} \gamma h, \\
r_h^* \Omega &= h^{-1} \Omega h.
\end{align*}
\end{lemma}
\begin{problem}{upstairs:prove.FLRG}
Take exterior derivative of both sides of \(0=d\omega_i + \gamma_{ij} \wedge \omega_j\), and apply Cartan's lemma to prove lemma~\vref{lemma:curvature.exists}.
\end{problem}
\begin{answer}{upstairs:prove.FLRG}
So far, we know that \(0=d\omega_i + \gamma_{ij} \wedge \omega_j\).
Take exterior derivative of both sides to arrive at
\[
0 = \pr{d\gamma_{ij} + \gamma_{ik} \wedge \gamma_{kj}} \wedge \omega_j.
\]
Apply Cartan's lemma to prove that there are \(1\)-forms \(\beta_{ijk}\) so that
\[
d\gamma_{ij} + \gamma_{ik} \wedge \gamma_{kj} = \beta_{ijk} \wedge \omega_k.
\]
These \(\beta_{ijk}\) are symmetric in \(j,k\).
Adding the above equation as given, together with the same equation with \(i\) and \(j\) swapped, we get
\[
0 = \pr{\beta_{ijk} + \beta_{jik}} \wedge \omega_k.
\]
By Cartan's lemma (lemma~\vref{lemma:Cartans}), 
\[
\beta_{ijk}+\beta_{jik} = f_{ijk\ell} \omega_{\ell},
\]
for unique functions \(f_{ijk\ell}\) symmetric in \(i,j\) and in \(k,\ell\).
Therefore, if we alternate using this equation and the symmetry of \(\beta_{ijk}\) in \(j,k\):
\begin{align*}
\beta_{ijk}
&=
-\beta_{jik} + f_{ijk\ell} \omega_{\ell},
\\
&=
-\beta_{jki} + f_{ijk\ell} \omega_{\ell},
\\
&=
\beta_{kji} + \pr{-f_{jki\ell} + f_{ijk\ell}} \omega_{\ell},
\\
&=
\beta_{kij} + \pr{-f_{jki\ell} + f_{ijk\ell}} \omega_{\ell},
\\
&=
-\beta_{ikj} + \pr{f_{kij\ell}-f_{jki\ell} + f_{ijk\ell}} \omega_{\ell},
\\
&=
-\beta_{ijk} + \pr{f_{kij\ell}-f_{jki\ell} + f_{ijk\ell}} \omega_{\ell},
\\
\intertext{so we add \(\beta_{ijk}\) to both sides and divide by 2,}
&=
\frac{1}{2}\pr{f_{kij\ell}-f_{jki\ell} + f_{ijk\ell}} \omega_{\ell}.
\end{align*}
Therefore
\[
d\gamma_{ij} + \gamma_{ik} \wedge \gamma_{kj} =
\frac{1}{2} R_{ijk\ell} \omega_k \wedge \omega_{\ell},
\]
where
\[
R_{ijk\ell}\defeq \frac{1}{2} 
\pr{
f_{kij\ell}-f_{jki\ell}
-
f_{\ell ijk}-f_{j\ell ik}
}.
\]
We leave the reader to check the identities.
\end{answer} 
The \emph{curvature tensor}\define{curvature!tensor} \(R\) is the object that associates to any 4 tangent vectors \(u,v,w,z \in T_m M\) the number
\[
\ip{R(u,v)w}{z} = R_{ijk\ell} u_i v_j w_k z_{\ell},
\]
where \(u=u_i e_i, v=v_i e_i, w=w_i e_i, z=z_i e_i\) in some orthonormal basis \(e_1, e_2, \dots, e_n\), and \(R_{ijk\ell}\) are the values of the function defined above at the frame \(e_1, e_2, \dots, e_n\).
One easily checks that
\[
r_h^* R_{ijk\ell} = R_{abcd} h_{ai} h_{jb} h_{kc} h_{\ell d},
\]
and this ensures that the curvature is well-defined.
Each smooth orthonormal framing is a smooth section of the frame bundle, in which we can write the curvature directly, and we see that the curvature \(\ip{R(u,v)w}{z}\) varies smoothly as we smoothly vary \(u,v,w,z\).


\section{Submanifolds}
Take an immersed submanifold \(P\) of a Riemannian manifold \(M\), where \(p\defeq \dim P\) and \(p+q \defeq \dim M\).
If we denote the immersion \(P \to M\) as \(f \colon P \to M\), a frame \((p_0,e_1,\dots,e_p,e_{p+1},\dots,e_{p+q})\) is \emph{adapted} if \(e_1, \dots, e_p \in f'(p_0) T_{p_0} P\).
Let \(\adaptedFrameBundle{P}{M}\) be the set of all adapted frames.
We have obvious maps \(\frameBundle{P} \gets \adaptedFrameBundle{P}{M} \to \frameBundle{M}\).
On \(\adaptedFrameBundle{P}{M}\), the \(1\)-forms \(\omega_i\) of \(\frameBundle{P}\) pull back to \(\omega_i\), while those of \(\frameBundle{M}\) either pullback to \(\omega_i\) if \(i=1,2,\dots,p\) or to \(0\) if \(i=p+1,p+2,\dots,p+q\).
It is convenient to let indices \(i,j,k,\ell\) vary over \(1,2,\dots,p\) and let \(a,b,c,d\) vary over \(p+1,p+2,\dots,p+q\), so \(\omega_I=0\) on \(\adaptedFrameBundle{P}{M}\).
The \emph{shape operator}\define{shape operator} is
\[
A \defeq \omega_i \gamma_{ia} e_a.
\]
\begin{lemma}
The shape operator of an immersed submanifold is a symmetric linear operator, pulled back from a unique symmetric linear operator on tangent vectors of \(P\), associating to each pair of tangent vectors to the submanifold a normal vector.
\end{lemma}
\begin{proof}
On \(\adaptedFrameBundle{P}{M}\), \(\omega_a=0\) so \(0=d\omega_a=-\gamma_{ai} \wedge \omega_i\).
By Cartan's lemma, \(\gamma_{ai}=-a_{aij} \omega_j\) for unique functions \(a_{aij}=a_{aji}\).
\end{proof}



%\section{Geodesic flow}
%Take a point \((m,e,y) \in \frameBundle{M} \times \R{n}\).
%For any orthogonal \(n \times n\) matrix \(h\), write \(r_h(m,e,y)\) to mean \((m,eh,\transpose{h} y)\).
%\begin{lemma}\label{lemma:quotient.TM}
%The map \((m,e,y) \in \frameBundle{M} \times \R{n} \mapsto y_i e_i \in TM\) is a fiber bundle map, invariant under the action of the orthogonal group \(\Orth{n}\), and the quotient space \(\frameBundle{M}/\Orth{n}\) is the tangent bundle \(TM\).
%\end{lemma}
%\begin{proof}
%Clearly \((m,eh,\transpose{h}y) \mapsto h_{ji} y_j h_{ki} e_k = y_i e_i\), so the map is invariant.
%The map is clearly smooth and onto, with fibers precisely the orbit of \(\Orth{n}\).
%In any local coordinates \(x_1,x_2,\dots,x_n\) on \(M\).
%Pick a orthonormal framing \(X_1(x),\dots,X_n(x)\).
%Take coordinates \(x_1,x_2,\dots,x_n,v_1,v_2,\dots,v_n\) on \(TM\), where each tangent vector is uniquely represented as
%\(v_i X_i(x)\).
%Write out coordinates \(x_1,\dots,x_n,h_{11},\dots,h_{nn}\) on \(\frameBundle{M}\) as above, and then our map is expressed in these coordinates as
%\[
%(x,h,y) \mapsto (x,v)
%\]
%where \(v_i = y_j h_{ji}\).
%So our map is a fiber bundle map.
%\end{proof}
%
%
%Define a vector field \(\geodesicVectorField\) on \(\frameBundle{M} \times \R{n}\) by the linear equations
%\begin{align*}
%\geodesicVectorField \hook \omega &= y, \\
%\geodesicVectorField \hook \gamma &= 0, \\
%\geodesicVectorField \hook dy &= 0.
%\end{align*}
%\begin{problem}{upstairs:descend}
%Prove that the vector field \(\geodesicVectorField\) is invariant under the orthogonal group action.
%Prove that there is a unique vector field, the \emph{geodesic spray}, also denoted \(\geodesicVectorField\), on the tangent bundle, to which \(\geodesicVectorField\) projects.
%\end{problem}
%\begin{answer}{upstairs:descend}
%By definition,
%\[
%\left.
%\geodesicVectorField \hook
%\begin{pmatrix}
%\omega \\
%\gamma \\
%dy
%\end{pmatrix}
%\right|_{(m,e,y)}
%=
%\begin{pmatrix}
%y \\
%0 \\
%0
%\end{pmatrix}.
%\]
%Therefore
%\begin{align*}
%r_{h*} \geodesicVectorField
%\hook
%\left.
%\begin{pmatrix}
%\omega \\
%\gamma \\
%dy
%\end{pmatrix}
%\right|_{(m,eh,\transpose{h}y)}
%&=
%\geodesicVectorField
%\hook
%r_h^*
%\left.
%\begin{pmatrix}
%\omega \\
%\gamma \\
%dy
%\end{pmatrix}
%\right|_{(m,e,y)},
%\\
%&=
%\geodesicVectorField
%\hook
%\left.
%\begin{pmatrix}
%\transpose{h} \omega \\
%\transpose{h} \gamma h \\
%\transpose{h} dy
%\end{pmatrix}
%\right|_{(m,e,y)},
%\\
%&=
%\geodesicVectorField
%\hook
%\left.
%\begin{pmatrix}
%\transpose{h} y \\
%0 \\
%0
%\end{pmatrix}
%\right|_{(m,e,y)},
%\\
%&=
%\geodesicVectorField
%\hook
%\left.
%\begin{pmatrix}
%\omega \\
%\gamma \\
%dy
%\end{pmatrix}
%\right|_{(m,eh,\transpose{h}y)},
%\end{align*}
%so \(r_{h*} \geodesicVectorField = \geodesicVectorField\).
%In our local coordinates, as in lemma~\vref{lemma:quotient.TM}, we see that \(\geodesicVectorField\) descends to a unique vector field on \(TM\).
%\end{answer}
%Take a flow line of \(\geodesicVectorField\) on \(TM\); its projection to \(M\) is a \emph{geodesic}.\define{geodesic}
%\begin{lemma}
%Every geodesic has constant speed.
%\end{lemma}
%\begin{proof}
%Let \(\pi \colon (m,e,y) \in \frameBundle{M} \times \R{n} \mapsto m \in M\).
%We take the curve \(m(t)=\pi \circ (m(t),e(t),y(t))\) associated to a flow line of \(\geodesicVectorField\).
%Clearly \(y\) is constant along the flow of \(\geodesicVectorField\) and \((\dot{m}(t),\dot{e}(t),0) \hook \omega = y\).
%But \(\omega\) is by definition \(\omega^i=g(e_i,\dot{m}(t))\), so \(\dot{m}(t)=y_i e_i\) and so \(|\dot{m}(t)|=|y|\) is constant.
%\end{proof}
%\begin{lemma}
%Pick a point \(m_0 \in M\) and \(v_0 \in T_{m_0} M\) on a Riemannian manifold \(M\).
%Let \(G\) be the group of all diffeomorphisms of \(M\) preserving the Riemannian metric and also preserving the point \(m_0\) and the vector \(v_0\).
%Then the geodesic flow line through \((m_0,v_0) \in T_{m_0} M\), i.e. the flow line of \(\geodesicVectorField\), is invariant under \(G\).
%\end{lemma}
%\begin{proof}
%Each diffeomorphism \(\phi\) of \(M\) preserving the Riemannian metric acts on \(\frameBundle{M}\) by taking 
%\[
%(m,e_1,\dots,e_n) \mapsto (\phi(m),\phi'(m)e_1,\dots,\phi'(m)e_n).
%\]
%Since the vector field \(\geodesicVectorField\) is defined in terms of the Riemannian metric, it is preserved by all such diffeomorphisms acting on \(\frameBundle{M}\).
%\end{proof}
%\begin{example}
%Euclidean space has lines as geodesics: each line is the set of fixed points of the group \(G\) of rigid motions preserving a point of the line and a nonzero vector tangent to the line at that point. 
%\end{example}
%\begin{example}
%Picking a plane, we can rotate the unit sphere by the rotations which fix every point of that plane.
%The group \(G\) of those rotations fixes the great circle on which the sphere intersects the plane, and fixes every tangent vector to that great circle.
%The group \(G\) then moves the unit vectors perpendicular to that plane arbitrarily.
%The sphere has great circles as geodesics: each great circle is the set of fixed points of the group \(G\) of rigid motions preserving a point of the great circle and a nonzero vector tangent to the great circle at that point.
%\end{example}
%\begin{example}
%Products of spheres: each geodesic projects to a great circle on each sphere.
%Conversely, pick one great circle on each sphere, and a path which winds at constant speed around each of those great circles.
%Moving along those paths simultaneously gives a geodesic.
%The product of the great circles is a flat torus, and the sphere's geodesics somewhere tangent to that flat torus are the geodesics of that flat torus.
%\end{example}
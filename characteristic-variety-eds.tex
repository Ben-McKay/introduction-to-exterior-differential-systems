\chapter{The characteristic variety}
\chapterSummary{We give a geometric description of the characteristics of the associated partial differential equations of exterior differential systems.}
\section{Linearization}\label{section:linearization}\SubIndex{linearization}
The reader unfamiliar with linearization of partial differential equations, or characteristics, might look at appendix~\ref{chapter:characteristics}.
Take an exterior differential system \(\II\) on a manifold \(M\), and an integral manifold \(X\).
Suppose that the flow of a vector field \(v\) on \(M\) moves \(X\) through a family of integral manifolds.
The tangent spaces of \(X\) are carried by the flow of \(v\) through integral elements of \(\II\).
Equivalently, the flow pulls back each form in \(\II\) to vanish on \(X\).
So \(0=\left.\LieDer_v \vartheta\right|_X\) for any \(\vartheta \in \II\).
\prob{vector.fields}{Prove that all vector fields \(v\) tangent to \(X\) satisfy this equation.}

More generally, suppose that \(E \subset T_m M\) is an integral element of \(\II\).
If a vector field \(v\) on \(M\) carries \(E\) through a family of integral elements, then \(0=\left.\LieDer_v \vartheta\right|_E\) for each \(\vartheta \in \II\).
\prob{linearize.coords}{Compute \(\left.\LieDer_v \vartheta\right|_E\) in coordinates.}
\begin{answer}{linearize.coords}
Take local coordinates \(x^1,x^2,\dots,x^p, y^1,y^2,\dots,y^q\), where \(E\) is the graph of \(dy=0\). 
If \(\vartheta = c_{IA} dx^I\wedge dy^A\), we will see that
\[
\left.\LieDer_v \vartheta\right|_E = 
\pderiv{v^a}{x^j} c_{Ia} dx^{Ij} + v^a \pderiv{c_I}{y^a} dx^I .
\]
Let \((-1)^I\) mean \((-1)^m\) if \(I\) consists of \(m\) indices.
Write
\[
v = v^j \pderiv{}{x^j} + v^b \pderiv{}{y^b}.
\]
Note that
\[
v \hook c_I dx^I = (-1)^J v^i c_{JiK} dx^{JK}.
\]
Commuting with exterior derivative,
\begin{align*}
\LieDer_v dx^i &= \pderiv{v^i}{x^j} dx^j + \pderiv{v^i}{y^b} dy^b, \\
\LieDer_v dy^a &= \pderiv{v^a}{x^j} dx^j + \pderiv{v^a}{y^b} dy^b.
\end{align*}
By the Leibnitz rule,
\begin{align*}
\LieDer_v \vartheta
&=
v^i \pderiv{c_{IA}}{x^i} dx^I \wedge dy^A 
+
v^a \pderiv{c_{IA}}{y^a} dx^I \wedge dy^A 
\\
&\qquad
+ c_{JiKA} dx^J \wedge \pr{\pderiv{v^i}{x^j} dx^j + \pderiv{v^i}{y^b} dy^b}  \wedge dx^K \wedge dy^A
\\
&\qquad
+ c_{IBaC} dx^I \wedge dy^B \wedge \pr{\pderiv{v^a}{x^j} dx^j + \pderiv{v^a}{y^b} dy^b}  \wedge dy^C.
\end{align*}
On \(E\), \(dy=0\) so
\begin{align*}
\left.\LieDer_v \vartheta\right|_E
&=
v^i \pderiv{c_I}{x^i} dx^I
+
v^a \pderiv{c_I}{y^a} dx^I
\\
&\qquad
+ c_{JiK} dx^J \wedge \pderiv{v^i}{x^j} dx^j  \wedge dx^K
\\
&\qquad
+ c_{Ia} dx^I \wedge \pderiv{v^a}{x^j} dx^j.
\end{align*}
This is not quite the same as
\begin{align*}
\left.v\hook d\vartheta\right|_E
&=
v^i\pderiv{c_I}{x^i}dx^I
+
v^a\pderiv{c_I}{y^a}dx^I
\\
&\qquad+(-1)^{jI}v^{\ell}\pderiv{c_{I\ell k}}{x^j}dx^{jIK}.
\end{align*}
Write the tangent part of \(v\) as
\[
v' = v^i \pderiv{}{x^i}.
\]
Let \(A\) be the linear map \(A \colon E \to E\) given by
\[
A^i_j = \pderiv{v^i}{x^j}
\]
and apply this by derivation to forms on \(E\) as
\[
(A\xi)(v_1,\dots,v_k)=\xi(Av_1,v_2,\dots,v_k)-\xi(v_1,Av_2,v_3,\dots,nv_k)+\dots.
\]
Then
\[
\left.\LieDer_v \vartheta\right|_E
=
\left.v' \hook d\vartheta\right|_E 
+
v^a \pderiv{c_I}{y^a} dx^I
+ \left.A\vartheta\right|_E
+ c_{Ia} dx^I \wedge \pderiv{v^a}{x^j} dx^j.
\]
Since \(0=\left.\vartheta\right|_E=\left.d\vartheta\right|_E\), we find
\[
\left.\LieDer_v \vartheta\right|_E
=
v^a \pderiv{c_I}{y^a} dx^I
+ c_{Ia} dx^I \wedge \pderiv{v^a}{x^j} dx^j.
\]%
\end{answer}
Take any submanifold \(X\).
Suppose that a differential form \(\vartheta\) vanishes on the linear subspace \(E\defeq T_m X\).
Writing \(X\) as the graph of some functions, the expression \(\left.\vartheta\right|_X\), as a nonlinear first order differential operator on those functions, has linearization  \(\vartheta \mapsto \left.\LieDer_v \vartheta\right|_E\).
That linearization is applied to sections \(v\) of the normal bundle \(\left.TM\right|_X/TX\).
\begin{problem}{proof.Cartan.Kaehler:choice.of.X}
In coordinates, prove that the linearized operator at the origin of our coordinates depends only on the integral element \(E=T_m X\), not on the choice of submanifold \(X\).
\end{problem}
\begin{answer}{proof.Cartan.Kaehler:choice.of.X}
Take local coordinates \(x^1,x^2,\dots,x^p, y^1,y^2,\dots,y^q\), where \(E\) is the graph of \(dy=0\). 
Write \(\vartheta = c_{IA} dx^I\wedge dy^A\).
Write \(X\) as the graph \(y=y(x)\), so
\[
\left.\vartheta\right|_X = \sum c_{IA}(x,y) dx^I \wedge 
\pderiv{y^{a_1}}{x^{j_1}} dx^{j_1} \wedge \dots \wedge \pderiv{y^{a_{\ell}}}{x^{j_{\ell}}} dx^{j_{\ell}}.
\]
so
\[
\left.\LieDer_v \vartheta\right|_E
=
v^a \pderiv{c_I}{y^a} dx^I
+ c_{Ia} dx^I \wedge \pderiv{v^a}{x^j} dx^j.
\]%
depends only on knowledge of the point \(m\) where we compute coefficients of \(\vartheta\) and of \(E=T_m X\), so that we drop \(dy\) terms.
\end{answer}
Linearize any exterior differential system about any integral element by linearizing its differential forms, i.e. by linearizing the differential equations given by asking that those forms vanish on submanifolds.
The linearization at a \(p\)-dimensional integral element depends only on \(\II^p_m\): the set of all values \(\vartheta_m\) of forms in \(\II^p\).
\begin{problem}{proof.Cartan.Kaehler:drop.P}
If \(P\defeq(E^{\perp})\subset \Lm{*}{T_m^*M}\), identify the linearized exterior differential system with \(\pr{\II_m + P^2}/P^2 \subset P/P^2 = E^{\perp} \otimes \Lm{*}{E}^*\).
\end{problem}
\begin{answer}{proof.Cartan.Kaehler:drop.P}
In coordinates, the linearization of an exterior differential system \(\II\) about an integral element \(E \subset T_m M\) does not involve any \(dy \wedge dy\) terms but depends on the terms with no \(dy\) and with one \(dy\) explicitly.
But \(P=\spn{E^{\perp}}=\spn{dy^a}\).
\end{answer}
In a tableau, \(P\) is generated by the polars, so the linearization is precisely the same tableau, modulo the nonlinearity, as in problem~\vref{problem:tableau:what.is.tableau}.
So the linearization about an involutive integral element is involutive.
On the other hand, the linearization at a noninvolutive integral element may or may not be involutive.
\prob{linearize}{Compute the linearization of \(u_{xx}=u_{yy} + u_{zz} + u_x^2\) around \(u=0\) by setting up this equation as an exterior differential system.}
\prob{linearize.vanishing}{For a section \(v\) of the normal bundle of \(X\) which vanishes at a point \(m\), compute the linearization \(\left.\LieDer_v \vartheta\right|_E\).}
\begin{answer}{linearize.vanishing}
\[
\left.\LieDer_v \vartheta\right|_E = c_{Ia} dx^I \wedge \pderiv{v^a}{x^j} dx^j
\]
\end{answer}

\section{The symbol}
\begin{problem}{char.var:symbol.matrix}
Compute the symbol of the linearization of a differential form \(\vartheta\) about an integral element \(E\) of that form, to find
\[
\opsymbol[]{\xi}v=\left.\xi\wedge(v\hook\vartheta)\right|_E.
\]
\end{problem}
\begin{answer}{char.var:symbol.matrix}
\begin{align*}
\opsymbol[]{df}v
&=
\lim_{\lambda\to\infty}
\frac{e^{-i\lambda f} \LieDer_{e^{i\lambda f}v}\vartheta}{i\lambda},
\\
&=
\lim_{\lambda\to\infty}
e^{-i\lambda f}
\frac{\left(d\left(e^{i\lambda f}v\hook\vartheta\right)+e^{i\lambda f}v\hook d\vartheta\right)}{i\lambda},
\\
&=
\lim_{\lambda\to\infty}
e^{-i\lambda f}
\frac{\left(i\lambda df\, e^{i\lambda f}\wedge v\hook\vartheta
+
e^{i\lambda f}d(v\hook\vartheta)+v\hook d\vartheta\right)}{i\lambda},
\\
&=
\lim_{\lambda\to\infty}\frac{\left(i\lambda df\wedge v\hook\vartheta
+
d(v\hook\vartheta)+v\hook d\vartheta\right)}{i\lambda},
\\
&=
df\wedge v\hook\vartheta
\end{align*}
This holds for any \(df\), so for any linear combinations of such, so for any \(1\)-form \(\xi\).
In our coordinates above, we compute the symbol of the associated system of partial differential equations by replacing \(\pderiv{v^{\alpha}}{x^i}\) by \(v^{\alpha}\xi_i\).
We immediately see that it agrees with this result: \(\xi\wedge v\hook\vartheta\).
\end{answer}
The same works with a column \(\vartheta\) of differential forms arising in a tableau for an exterior differential system.
To compute the symbol from a tableau:
\begin{enumerate}
\item
Drop the nonlinearity.
\item
Turn each polar \(\pi^{\alpha}\) into a formal expression \(v^{\alpha}\xi\) where \(\xi=\xi_i\omega^i\).
\item
Expand out the tableau, and collect up terms in each \(\omega^I\).
\item
Write out these terms, each a linear expression in the variables \(v^{\alpha}\), as a product of a matrix row with a vector of components of \(v\): the symbol \(\opsymbol[]{\xi}v\).
\end{enumerate}
To see this, compute \(\left.\xi\wedge(v\hook\vartheta)\right|_E\) by plugging in \(v\) to each polar \(\pi^{\alpha}\), yielding \(v^{\alpha}\), and wedge in a factor of \(\xi\).
Danger: in the recipe above, we are missing a part of the symbol: the \(\theta^a\)-components \(v^a\) of \(v\) also show up in \(\left.\xi\wedge v\hook\vartheta\right|_E\), in expressions 
\[
\left.\xi\wedge v\hook\theta^a\right|_E=v^a\xi=v^a\xi_i\omega^i.
\]
Each yields an expression \(v^a\xi_i\), giving a row to the symbol matrix, with one nonzero entry \(\xi_i\) in column \(a\), for all \(i,a\).
We will ignore these rows in our computations because, as we will see, they make a trivial and predictable contribution to the symbol and the characteristic variety, so we can just assume that \(v^a=0\) for each \(\theta^a\).
\begin{problem}{char.var:calc.symbol}
Find the symbol of the prolonged isometric embedding exterior differential system from chapter~\ref{chapter:prolongation}.
\end{problem}
\begin{answer}{char.var:calc.symbol}
Recall the tableau:
\[
d
\begin{pmatrix}
  \theta_4 \\
  \theta_5
\end{pmatrix}
=
-
\Tablo{*Da,Db,0;*Db,Dc,0}[2,0,0]
\wedge 
\begin{pmatrix}
  \pf_1 \\
  \pf_2 \\
  \alpha
\end{pmatrix}
\mod{\theta_1,\dots,\theta_6}
\]
(skipping zero rows) on the manifold \(M\) on which \(ac-b^2=K\) and not all of \(a,b,c\) are zero.
Since the matrix
\[
\begin{pmatrix}
a&b\\
b&c
\end{pmatrix}
\]
is symmetric and nonzero, and conjugated by rotation of frames, we arrange that \(a\ne 0\) at the point where we are working.
We find 
\[
Dc=-\frac{c}{a}Da+\frac{2b}{a}Db.
\]
To fit our notation better, write \(Da,Db\) as \(\pi^1,\pi^2\) and \(\alpha\) as \(\omega_3\).
So the tableau is
\[
d
\begin{pmatrix}
  \theta^4 \\
  \theta^5
\end{pmatrix}
=
-
\Tablo{*\pi_1,\pi_2,0;*\pi_2,-\frac{c}{a}\pi_1+2\frac{b}{a}\pi_2,0}[2,0,0]
\wedge 
\begin{pmatrix}
  \pf^1 \\
  \pf^2 \\
  \pf^3
\end{pmatrix}
\mod{\theta^1,\dots,\theta^6},
\]
skipping zero rows.
Write a vector \(v\in T_m M/E\) with components \(V^a=\theta^a(v)\), written with a capital \(V\) to avoid confusion with components \(v^{\alpha}=\pi^{\alpha}(v)\).  
We compute the symbol for each of the forms
\[
\theta^1,\dots,\theta^6,d\theta^4,d\theta^5
\]
and then stack these symbols on top of one another as one big matrix, the symbol of the exterior differential system.
For an arbitrary \(\xi=\xi_1\omega^1+\xi_2\omega^2+\xi_3\omega^3\),
\[
\xi\wedge v\hook\theta^a=V^a\xi_1\omega^1+V^a\xi_2\omega^2+V^a\xi_3\omega^3,
\]
contributes rows for each of the six \(\theta^a\), one row for each \(\omega^i\), so \(6\cdot 3=18\) rows:
\[
\begin{pmatrix}
\xi_1 & 0 & 0 & 0 & 0 & 0 & 0 & 0 \\
\xi_2 & 0 & 0 & 0 & 0 & 0 & 0 & 0 \\
\xi_3 & 0 & 0 & 0 & 0 & 0 & 0 & 0 \\
0 & \xi_1 & 0 & 0 & 0 & 0 & 0 & 0 \\
0 & \xi_2 & 0 & 0 & 0 & 0 & 0 & 0 \\
0 & \xi_3 & 0 & 0 & 0 & 0 & 0 & 0 \\
0 & 0 & \xi_1 & 0 & 0 & 0 & 0 & 0 \\
0 & 0 & \xi_2 & 0 & 0 & 0 & 0 & 0 \\
0 & 0 & \xi_3 & 0 & 0 & 0 & 0 & 0 \\
0 & 0 & 0 & \xi_1 & 0 & 0 & 0 & 0 \\
0 & 0 & 0 & \xi_2 & 0 & 0 & 0 & 0 \\
0 & 0 & 0 & \xi_3 & 0 & 0 & 0 & 0 \\
0 & 0 & 0 & 0 & \xi_1 & 0 & 0 & 0 \\
0 & 0 & 0 & 0 & \xi_2 & 0 & 0 & 0 \\
0 & 0 & 0 & 0 & \xi_3 & 0 & 0 & 0 \\
0 & 0 & 0 & 0 & 0 & \xi_1 & 0 & 0 \\
0 & 0 & 0 & 0 & 0 & \xi_2 & 0 & 0 \\
0 & 0 & 0 & 0 & 0 & \xi_3 & 0 & 0
\end{pmatrix}
\begin{pmatrix}
V^1\\
V^2\\
V^3\\
V^4\\
V^5\\
V^6\\
v^1\\
v^2
\end{pmatrix}
\]
These are the trivial and predictable rows, as we said above.
(We also said that they can be ignored when we compute the characteristic variety.)
They are the only rows containing any \(V^a\) components.
(So all \(V^a\) components are irrelevant to the characteristic variety, since these rows will just force all \(V^a\) components to vanish, as we will see.)
The remaining rows come from working out the various \(\omega^i\wedge\omega^j\) components of \(\left.\xi\wedge v\hook d\theta^a\right|_E\), \(a=4,5,6\).
For example,
\begin{align*}
\xi\wedge v\hook d\theta^4
&=
(\xi_1\omega^1+\xi_2\omega^2+\xi_3\omega^3)
\wedge
(v^1\omega^1+v^2\omega^2),
\\
&=
(\xi_1v^2-\xi_2v^1)\omega^{12}+\xi_3v^1\omega^{31}+\xi_3v^2\omega^{32}.
\end{align*}
We add a row representing \(\xi_1v^2-\xi_2v^1\), i.e.
\[
\begin{pmatrix}
0& 0& 0& 0& 0& 0 & -\xi_2 & \xi_1 
\end{pmatrix}
\begin{pmatrix}
V^1\\
V^2\\
V^3\\
V^4\\
V^5\\
V^6\\
v^1\\
v^2
\end{pmatrix}
\]
We add two more rows representing \(\xi_3v^1,\xi_3v_2\):
\[
\begin{pmatrix}
0& 0& 0& 0& 0& 0 & \xi_3 & 0 \\
0& 0& 0& 0& 0& 0 & 0 & \xi_3
\end{pmatrix}
\]
Similarly, \(d\theta^5\) yields rows
\[
\begin{pmatrix}
0& 0& 0& 0& 0& 0 & -\frac{c}{a}\xi_1 & 2\frac{b}{a}\xi_1-\xi_2 \\
0& 0& 0& 0& 0& 0 & 0 & -\xi_3 \\
0& 0& 0& 0& 0& 0 & \frac{c}{a}\xi_3 & -2\frac{b}{a}\xi_3 \\
\end{pmatrix}
\]
Finally, the symbol is
\[
\sigma(\xi)=
\begin{pmatrix}
\xi_1 & 0 & 0 & 0 & 0 & 0 & 0 & 0 \\
\xi_2 & 0 & 0 & 0 & 0 & 0 & 0 & 0 \\
\xi_3 & 0 & 0 & 0 & 0 & 0 & 0 & 0 \\
0 & \xi_1 & 0 & 0 & 0 & 0 & 0 & 0 \\
0 & \xi_2 & 0 & 0 & 0 & 0 & 0 & 0 \\
0 & \xi_3 & 0 & 0 & 0 & 0 & 0 & 0 \\
0 & 0 & \xi_1 & 0 & 0 & 0 & 0 & 0 \\
0 & 0 & \xi_2 & 0 & 0 & 0 & 0 & 0 \\
0 & 0 & \xi_3 & 0 & 0 & 0 & 0 & 0 \\
0 & 0 & 0 & \xi_1 & 0 & 0 & 0 & 0 \\
0 & 0 & 0 & \xi_2 & 0 & 0 & 0 & 0 \\
0 & 0 & 0 & \xi_3 & 0 & 0 & 0 & 0 \\
0 & 0 & 0 & 0 & \xi_1 & 0 & 0 & 0 \\
0 & 0 & 0 & 0 & \xi_2 & 0 & 0 & 0 \\
0 & 0 & 0 & 0 & \xi_3 & 0 & 0 & 0 \\
0 & 0 & 0 & 0 & 0 & \xi_1 & 0 & 0 \\
0 & 0 & 0 & 0 & 0 & \xi_2 & 0 & 0 \\
0 & 0 & 0 & 0 & 0 & \xi_3 & 0 & 0 \\
0& 0& 0& 0& 0& 0 & -\xi_2 & \xi_1 \\
0& 0& 0& 0& 0& 0 & \xi_3 & 0 \\
0& 0& 0& 0& 0& 0 & 0 & \xi_3 \\
0& 0& 0& 0& 0& 0 & -\frac{c}{a}\xi_1 & 2\frac{b}{a}\xi_1-\xi_2\\
0& 0& 0& 0& 0& 0 & 0 & -\xi_3 \\
0& 0& 0& 0& 0& 0 & \frac{c}{a}\xi_3 & -2\frac{b}{a}\xi_3 \\
\end{pmatrix}.
\]
\end{answer}
More invariantly, without choosing any tableau, the symbol of an exterior differential system \(\II\) at an integral element \(E\subseteq T_m M\) is
\[
\opsymbol{} \colon \xi \in E^*, v \in T_m M/E, \vartheta\in\II_m^p \mapsto 
\left.\xi\wedge(v\hook\vartheta)\right|_E\in\Lm{p}{E}^*,
\]
so 
\[
\opsymbol{}\in E\otimes E^{\perp}\otimes\bigoplus_p\II_m^{p*}\otimes\Lm{p}{E}^*.
\]

\section{The characteristic variety}
An integral element is \emph{noncharacteristic}\define{characteristic} if it is a hyperplane in precisely one integral element.
An integral manifold is \emph{noncharacteristic} (or \emph{characteristic}) if all of its tangent spaces are.
A characteristic hypersurface in an integral manifold could perhaps lie in more than one integral manifold, a potential nontangential intersection of integral manifolds.
Glue along such an intersection, to create a ``crease'' along an integral manifold.
So we imagine that integral manifolds are more flexible along characteristic hypersurfaces, although there is no theorem to prove that.
\prob{char.var:wave}{Find the characteristics of the wave equation as an exterior differential system. 
Show creasing of solutions along characteristics, and not along noncharacteristic curves.}
\begin{lemma}\label{lemma:char.var}
Take an integral manifold \(X\) of an exterior differential system \(\II\), a point \(x \in X\), and let \(E\defeq T_x X\).
The characteristic variety\define{characteristic!variety}\define{variety!characteristic} 
\(
\CV_x \subset \Proj(E^*)
\)
of the linearization is the set of characteristic hyperplanes in \(E\).
\end{lemma}
\begin{proof}
The characteristic variety \(\CV_x\) is the set of hyperplanes \([\xi]=(\xi=0)\) associated to nonzero cotangent vectors \(\xi \in E^*\) for which there is some section \(v\) of the normal bundle of \(X\) with \(v(x)\ne 0\) and \(0=\opsymbol[]{\xi}v\).
\[
\opsymbol[]{\xi}(v)\vartheta=\xi \wedge (v \hook \vartheta)|_E.
\]
for every \(\vartheta \in \II^p_m\).
This says precisely that the vector \(v\) can be added to the hyperplane \(E\cap (0=\xi) \subset E\) to make an integral element enlarging the hyperplane.
\end{proof}
This lemma allows us to define the characteristic variety of any integral element, even if not tangent to any integral manifold: the \emph{characteristic variety} of an integral element \(E\) is the set of characteristic hyperplanes in \(E\), denoted \(\CV_E\subset \Proj(E^*)\).

If the symbol matrix has more columns than rows, then (linear algebra!) some such \(v\) exists for any \(\xi\): the characteristic variety consists of all hyperplanes in the integral element.
If the symbol matrix has at least as many rows than columns, then the determinants of the minors of the symbol matrix are the equations of \(\CV\), as the existence of such a section \(v\) is precisely the noninvertibility of any submatrix.
\begin{problem}{char.var:isom.cont}
Continue problem~\vref{problem:char.var:calc.symbol} by computing the characteristic variety of the prolonged isometric immersion system.
\end{problem}
\begin{answer}{char.var:isom.cont}
We can make a submatrix of the symbol, cutting out various rows, which has determinant \(\xi_3^8\), so \(\xi_3=0\) on the characteristic variety.
So plug in \(\xi_3=0\) to the symbol matrix and start again.
We can similarly pick out \(8\) rows whose determinant is
\[
\xi_1^a\xi_2^b
\det
\begin{pmatrix}
-\xi_2&\xi_1\\
p_1\xi_1&p_2\xi_1-\xi_2
\end{pmatrix}
\]
for any \(a+b=6\): pick any \(a\) rows which have \(\xi_1\) as their only nonzero entry, all in different columns, and similarly for \(\xi_2\).
We must have \(\xi\ne 0\), i.e. one of \(\xi_1,\xi_2,\xi_3\) is not zero.
We know \(\xi_3=0\), so one of \(\xi_1,\xi_2\) is not zero.
Therefore the vanishing of these determinants is precisely the vanishing of
\[
\det
\begin{pmatrix}
-\xi_2&\xi_1\\
-\frac{c}{a}\xi_1&2\frac{b}{a}\xi_1-\xi_2
\end{pmatrix}
=
\frac{1}{a}(c\xi_1^2-2b\xi_1\xi_2+a\xi_2^2).
\]
We have assumed that \(a\ne 0\).
So the characteristic variety is the set of hyperplanes \([\xi]\) for
\[
\xi=\xi_1\omega^1+\xi_2\omega^2+\xi_3\omega^3
\]
for which \(\xi_3=0\) and \(0=c\xi_1^2-2b\xi_1\xi_2+a\xi_2^2\).
The equation \(\xi_3=0\) cuts out a projective line in the projective plane \(\Proj(E^*)\), while the quadratic equation cuts out zero, one, or two points in that projective line, depending on the determinant of 
\[
\begin{pmatrix}
a&b\\
b&c
\end{pmatrix},
\]
the proposed shape operator of the surface associated to an integral manifold.
\end{answer}
Note: as in this problem, when we compute characteristic variety of a tableau, the peculiar \(1\times 1\) blocks \(v^a\xi_j\) of the symbol, occuring for every \(v^a\) and \(\xi_j\), can be dropped from the computation of \(\CV\), since they just put in products of all \(\xi_j\) into the determinants, and at least one of the \(\xi_j\) must be nonzero, so we can just assume that all \(v^a\) vanish for all \(\theta^a\), and simplify the symbol to omit those columns.
\prob{proof.Cartan.Kaehler}{For an involutive integral element, prove that the following are equivalent:
\begin{enumerate}
\item
the integral element contains a noncharacteristic hyperplane,
\item
every regular hyperplane is noncharacteristic,
\item
the final character is zero.
\end{enumerate}}
\begin{answer}{proof.Cartan.Kaehler}
Suppose \(E\subset E_+\) is a noncharacteristic hyperplane.
By uniqueness of extension \(E_+\), the polar equations of \(E\) cut out precisely \(E_+\) inside \(T_m M\), i.e. the dimension of polar equations of \(E\) is the dimension of \(T_m M/E_+\).
The polar equations of any regular hyperplane in \(E_+\) are satisfied on \(E_+\), so have same rank.
So the regular integral elements are also noncharacteristic.
If \(p\defeq\dim E_+\), the rank of polar equations of \(E\) is \(s_0+s_1+\dots+s_{p-1}\), while the dimension of \(T_m M/E_+\) is \(s_0+\dots+s_p\).
So \(s_p=0\) on \(E_+\) just when every regular hyperplane in \(E_+\) is noncharacteristic.
\end{answer}
\prob{include.eds}{More equations, fewer characteristics: if \(\JJ \subset \II\), prove that \(\CV_{\II}\subset \CV_{\JJ}\).}%
\begin{answer}{include.eds}%
Any hyperplane \(E_- \subset E\) is \(\II\)-characteristic just when it lies in some other \(\II\)-integral element \(E'\) of same dimension as \(E\). But then \(E'\) is also a \(\JJ\)-integral element, so \(E_-\) is \(\JJ\)-characteristic: \(\CV^{\II}_E\subset \CV^{\JJ}_E\).%
\end{answer}
\prob{proof.CK:Frob.char.var}{What is the characteristic variety of a Frobenius\SubIndex{Frobenius theorem}\SubIndex{theorem!Frobenius} exterior differential system?}
\begin{answer}{proof.CK:Frob.char.var}At each point, there is a unique maximal integral element, so every hyperplane in it lies in that unique maximal integral element: \(\CV_E\) is empty.
\end{answer}

\section{Determined systems}
A \(p\)-dimensional integral element \(E\subset T_{m_0} M\) of an exterior differential system \(\II\) is \emph{determined}\define{determined!integral element} if \(E\) contains a noncharacteristic hyperplane and \(\II^p_m\) has constant dimension \(s_0+s_1+\dots+s_{p-1}\) for \(m\) near \(m_0\).
Note that then \(s_p=0\).
\begin{theorem}\label{theorem:CKbaby}
Take an analytic exterior differential system.
Every noncharacteristic analytic integral manifold \(X\), whose every tangent space is a hyperplane in a determined integral element, is a hypersurface in an analytic integral manifold.
Any two analytic integral manifolds containing \(X\) as a hypersurface share a set, open in both, containing \(X\).
\end{theorem}
\begin{proof}
As above, the symbol \(\opsymbol[]{\xi}\) at each \(p\)-dimensional integral element \(E\) is a linear map in
\[
T_m M/E \to \II^{p*}_m \otimes \Lm{p}{E^*},
\]
square at each determined integral element \(E\).
The dimension of \(\II^p_m\) cannot drop below \(s_0+s_1+\dots+s_p\), because it generates that many polar equations at \(E\), and hence near \(E\).
Pick out that number of linearly independent \(p\)-forms from \(\II\): they span \(\II^p\) at every point nearby.
The determined exterior differential system they generate has the same \(p\)-dimensional integral manifolds, polar equations, and symbol.
\end{proof}
\prob{char.var:pseudo.hol}{Prove that every real immersed curve in an almost complex manifold lies in an immersed holomorphic disk.}
\prob{proof.test:char.var}{Prove that if \(E\subset E_+\) is characteristic, and \(E_+\) has dimension \(p\) then \(s_p>0\).}
\begin{answer}{proof.test:char.var}
The choices of \(p\)-dimensional integral element arise from the semipositive grade coefficients: \(k\) coefficients of each polar in grade \(k\), so \(ks_k\) in that grade in all.
So if \(s_p=0\) then there is a unique \(p\)-dimensional integral element containing \(E\).
If \(E_+\) is involutive, then the semipositive grade coefficients \(p^{\alpha}_i\) are arbitrary.
In an adapted tableau, \(E\) is determined by \(p^{\alpha}_i\) for \(i<p\), which are all of the coefficients just when \(s_p=0\).
\end{answer}
\begin{example}
For the tableau of triply orthogonal webs
\[
\begin{gradedTableau}{|c|cc}
\freeDeriv{\pi^3} & 0 & 0\\
0 & \freeDeriv{\pi^2} & 0 \\
0 & 0 & \freeDeriv{\pi^1} 
\end{gradedTableau}\wedge
\begin{gradedIndependents}
\emptyGrade
\omega^{12}\+
\omega^{13}\\ 
\omega^{23}
\end{gradedIndependents}
\]
we find
\begin{align*}
\begin{pmatrix}
v^3 & 0 & 0 \\
0 & v^2 & 0 \\
0 & 0 & v^1
\end{pmatrix}\xi\wedge
\begin{pmatrix}
\omega^{12}\\
\omega^{13}\\ 
\omega^{23}
\end{pmatrix}
&=
\begin{pmatrix}
v^3 & 0 & 0 \\
0 & v^2 & 0 \\
0 & 0 & v^1
\end{pmatrix}
\begin{pmatrix}
\xi_3\\
-\xi_2\\ 
\xi_1
\end{pmatrix}
\omega^{123},\\
&=\begin{pmatrix}
0 & 0 & \xi_3 \\
0 & -\xi_2 & 0 \\
\xi_1 & 0 & 0
\end{pmatrix}
\begin{pmatrix}
v^1\\
v^2\\
v^3
\end{pmatrix}\omega^{123},
\\
&=\opsymbol[]{\xi}v\omega^{123}.
\end{align*}
So \(\det\opsymbol[]{\xi}=\xi_1\xi_2\xi_3\): the characteristic variety \(\CV\) is the triple of lines \((\xi_1=0)\), \((\xi_2=0)\) and \((\xi_3=0)\).
Any \(3\)-dimensional integral element coframed by \(\omega^1,\omega^2,\omega^3\) has characteristic hyperplanes \(0=\xi_1\omega^1+\xi_2\omega^2+\xi_3\omega^3\) for any one of the three coefficients vanishing, i.e. containing any one of the three axes.
\begin{theorem}
Take an embedded analytic surface \(S \subset \E[3]\).
Pick orthonormal analytic vector fields \(e_1,e_2,e_3\) along \(S\), none tangent to \(S\).
Then there is a triply orthogonal web near \(S\) in \(\E[3]\) with leaves perpendicular to \(e_1,e_2,e_3\) at each point of \(S\).
Any two such agree near \(S\).
\end{theorem}
The reader familiar with Riemannian geometry may recognize that this proof works identically replacing \(\E[3]\) by any analytic Riemannian \(3\)-manifold.
Any symmetry of \(S\) and \(e_1,e_2,e_3\) on \(S\) is shared by the triply orthogonal web, by uniqueness.
\end{example}
\begin{example}
In proving Lie's third theorem,\SubIndex{Lie's third theorem}\SubIndex{theorem!Lie's third} we had tableau
\[
\begin{pmatrix}
\freeDeriv{\pi^1_1} & \freeDeriv{\pi^1_2} & \dots & \freeDeriv{\pi^1_p} \\
\freeDeriv{\pi^2_1} & \freeDeriv{\pi^2_2} & \dots & \freeDeriv{\pi^2_p} \\
\vdots & \vdots & \dots & \vdots \\
\freeDeriv{\pi^p_1} & \freeDeriv{\pi^p_2} & \dots & \freeDeriv{\pi^p_p}
\end{pmatrix}
\wedge
\begin{pmatrix}
\omega^1 \\
\omega^2 \\
\vdots\\
\omega^p
\end{pmatrix},
\]
i.e. \(0=\freeDeriv{\pi^i_j} \wedge \omega^j\).
We plug in \(\freeDeriv{\pi^i_j} = v^{\alpha}\xi\), to get
\[
0=\frac{1}{2} (v^i_j \xi_k-v^i_k \xi_j)\omega^{kj}.
\]
Our linear expressions are \(v^i_j \xi_k - v^i_k \xi_j\).
One particular solution \(v\) to these equations is \(v^i_j\defeq \xi_j\) for all \(i\).
So every hyperplane in every \(p\)-dimensional integral element is characteristic.
This is not surprising: any diffeomorphism takes every Maurer--Cartan form to a Maurer--Cartan form; we can pick a diffeomorphism which is very close to the identity except along a hypersurface ``bending'' every Maurer--Cartan form ``along'' that hypersurface.
\end{example}
\prob{char.var:nabla.char}{Find the characteristic variety of \(\nabla \times u = u - f\).}
\begin{answer}{char.var:nabla.char}
In terms of the tableau we gave previously in solving problem~\vref{problem:proof.test:nabla}, the complex points in the projective plane satisfying the equations of the minors are
\[
[0,1,0], [i,1,0], [-i,1,0], [i,0,1], [-i,0,1], [1,1,\sqrt{2}i], [1,1,-\sqrt{2}i].
\]
The real ones for the characteristic variety as defined above, i.e. just \([0,1,0]\) corresponding to the hyperplane \(0=dx^2\).
\end{answer}
\begin{problem}{char.var:harmonic}
Find the characteristic variety for the exterior differential system of harmonic functions in the plane.
Show that it doesn't satisfy the conditions of the Cauchy--Kovalevskaya theorem as described in theorem~\vref{theorem:CKbaby}.
(Harder: can you ``fix it'', i.e. use the Cauchy--Kovalevskaya theorem to nonetheless prove the existence of local integral surfaces?)
\end{problem}
\begin{answer}{char.var:harmonic}
For the harmonic function exterior differential system \(\II\),
\[
\II^2_{(x,y,u_x,u_y)}
=
\spn{\Theta,d\vartheta,\vartheta\wedge(dx,dy,du_x,du_y)}
\]
has dimension 6.
The characters are \(s_0,s_1,s_2=1,2,0\), summing to \(3\), as we see from the tableau
\[
\Tablo{*du_y,!-du_x;*du_x,du_y}
\wedge
\begin{gradedIndependents}
dx\+
dy
\end{gradedIndependents}
\]
So the symbol is not square: \(s_0+s_1+s_2=3<6=\dim\II^2_{(x,y,u_x,u_y)}\).

Here is a way to ``fix'' this problem.
Let \(\II'\) be the ideal generated by the \(2\)-forms \(\vartheta\wedge dx,\Theta,d\vartheta\).
The tableau of \(\II'\) is
\[
\Tablo{
*\vartheta,0;*du_y,!-du_x;*du_x,du_y}
\wedge
\begin{gradedIndependents}
dx\+
dy
\end{gradedIndependents}
\]
with characters \(s_0',s_1',s_2'=0,3,0\).
But now 
\[
{\II'}^2_{(x,y,u_x,u_y)}
=
\spn{\Theta,d\vartheta,\vartheta\wedge dx}
\]
has dimension \(3\).
Take an \(\II'\)-integral surface \(S\) coframed by \(dx,dy\) and containing an \(\II\)-integral curve \(C\) on which \(dx\ne 0\), say with \(y=y(x)\).
On \(S\), \(\vartheta\wedge dx=0\) so \(\vartheta=f(x,y)\,dx\) locally.
On \(C\), \(0=\vartheta=f(x,y(x))\,dx\).
So \(f(x,y(x))=0\).
On \(S\), \(0=d\vartheta=df\wedge dx\), so \(f=f(x)\) and \(f(x,y)=f(x)=f(x,y(x))=0\), hence \(\vartheta=0\) on \(S\).
We conclude that any \(\II\)-integral curve \(C\) on which \(dx\ne 0\) lies in a locally unique integral surface \(S\).
The problem with directly applying the Cauchy--Kovalevskaya theorem is that \(\II\)-integral curves are not arbitrary curves.
\end{answer}

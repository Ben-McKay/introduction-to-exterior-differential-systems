\chapter{Analytic convergence}
\section{Convergence}\label{section:convergence}
A sequence \(f_1,f_2,\dots\) of analytic functions of real variables, defined on some open set, \emph{converges}\define{analytic!convergence}\define{convergence!analytic} to an analytic function \(f\) just when that set lies in an open set in complex variables to which \(f\) and all but finitely many \(f_i\) extend and where \(f_i\) converge uniformly to \(f\) on compact sets.
By the Cauchy integral theorem \cite{Ahlfors:1978} p. 120, convergence is uniform with all derivatives on compact sets.
\begin{problem}{Cauchy.Kov:Martineau}Prove: a sequence \(f_1,f_2,\dots\) of analytic functions converges to an analytic function \(f\) just when we can cover their domain in compact sets, and cover each compact set by an open set in complex variables, to which \(f\) and all but finitely many \(f_i\) extend and are bounded, and on which \(f_i\) converge uniformly to \(f\).
\end{problem}

\section{Commutative algebra}
A \emph{graded ring}\define{graded!ring} is a ring \(R\) which is the direct sum of abelian groups \(R_0,R_1,\dots\) so that \(R_iR_j\subseteq R_{i+j}\), \emph{elementary}\define{elementary!graded ring}\define{graded!ring!elementary} if generated by \(R_0\) and \(R_1\).
An \(R\)-module \(M\) is \emph{graded}\define{graded!module} if a direct sum of abelian groups \(M_i\) with \(R_iM_j\subseteq M_{i+j}\); \emph{asymptotically simple}\define{asymptotically simple}\define{graded!module!asymptotically simple} if \(R_1M_k=M_{k+1}\) for all sufficiently large \(k\); \emph{elementary}\define{graded!module!elementary} if each \(M_k\) is finitely generated as an \(R_0\)-module.
\begin{lemma}%
[Phat Nguyen \cite{PhatNguyen:2014}]%
\label{lemma:Phat}%
\define{lemma!Phat Nguyen}\define{Phat Nguyen lemma}
An elementary graded module over an elementary graded ring is finitely generated if and only if asymptotically simple.
\end{lemma}
\begin{proof}
Suppose \(M\) is finitely generated over \(R\), hence by \(M_0+M_1+\dots+M_k\).
For any \(j \ge k\), 
\[
M_{j+1}=R_{j+1}M_0+R_jM_1+\dots+R_1M_j.
\]
Since \(R\) is elementary
\[
M_{j+1}=R_1^{j+1}M_0+R_1^jM_1+\dots+R_1M_j\subseteq R_1M_j.
\]
But \(R_1M_j\subseteq M_{j+1}\), so \(M_{j+1}=R_1M_j\) for all \(j\ge k\).
Therefore \(M\) is asymptotically simple.

Conversely, assume  \(M\) asymptotically simple: \(R_1M_j=M_{j+1}\) for any \(j\ge k\). 
As an \(R\)-module, \(M\) is generated by \(M_0+M_1+\dots+M_k\).
Because \(M\) is elementary, we can select a finite set of generators from each of \(M_0,\dots,M_k\) as an \(R_0\)-module, hence as an \(R\)-module.
\end{proof}
A module \(M\) over a ring \(R\) is \emph{Noetherian}\define{Noetherian} if every \(R\)-submodule is finitely generated; \(R\) is \emph{Noetherian} if every ideal is finitely generated.
\begin{theorem}%
[Krull Intersection Theorem \cite{PhatNguyen:2014}]%
\label{theorem:Krull.intersection}%
\define{Krull intersection theorem}%
\define{theorem!Krull intersection}%
Suppose that \(R\) is a Noetherian commutative ring with identity, \(I\subseteq R\) an ideal, \(M\) a finitely generated \(R\)-module, and 
\[
N\defeq\bigcap_{j=0}^{\infty} I^j M.
\]
Then there is an element \(i\in I\) so that \((1+i)N=0\).
If all elements of \(1+I\) are units of \(R\), then \(N=0\).
\end{theorem}
\begin{proof}
Write \(R,I,M,N\) as \(R_0,I_0,M_0,N_0\).
Define elementary graded ring and modules:
\[
\begin{array}{r@{\,}c@{\,}l@{\,}c@{\,}l@{\,}c@{\,}l@{\,}c@{\,}c}
R&\defeq&R_0&\oplus&I_0&\oplus&I_0^2&\oplus&\dots,\\
M&\defeq&M_0&\oplus&I_0M_0&\oplus&I_0^2M_0&\oplus&\dots,\\
N&\defeq&N_0&\oplus&N_0&\oplus&N_0&\oplus&\dots.
\end{array}
\]
Since \(M\) is asymptotically simple, \(M\)  is  finitely  generated  over \(R\).
Since \(R_0\) is Noetherian, \(R\) is Noetherian, so \(M\) is Noetherian as an \(R\)-module.
Because \(N_0\subset I_0^jM_0\)  for  \(j=1,2,\dots\), \(N\) is a graded  \(R\)-submodule  of \(M\),  so finitely generated over \(R\).  By lemma~\vref{lemma:Phat}, \(N\) is asymptotically simple, so \(I_0N_0 = N_0\).

Take generators \(n_a\in N_0\).
Then \(n_a=\sum_b j_{ab}n_b\) for some \(j_{ab} \in I_0\).
Let \(J=(j_{ab})\), a matrix, and \(n=(n_a)\) a vector.
So \((I-J)n=0\).
Multiply by the adjugate matrix (the transpose of the matrix of cofactors) of \(I-J\) to get \(\det(I-J)n=0\); set \(1+i=\det(I-J)\).
\end{proof}


\section{Germs}
The \emph{germ}\define{germ} of an analytic function at a point is its equivalence class, two functions being equivalent if they agree near that point.
A sequence of germs \emph{converges}\define{germ!convergence}\define{convergence!germ} just when it arises from a convergent sequence of analytic functions.
By majorizing, this is just when the Taylor coefficients converge to those of an analytic function.
Germs form a ring, under addition and multiplication of functions.
Pick a finite dimensional real vector space.
Germs of analytic maps to that vector space form a module over that ring.
\begin{theorem}[Hilbert basis theorem \cite{Demailly:2012} p. 81 theorem 2.7, \cite{Hoermander:1990} p. 161 lemma 6.3.2, theorem 6.3.3, \cite{Taylor:2002} p. 44 theorem 3.4.1]\label{theorem:Hilbert.basis}\define{Hilbert!basis theorem}\define{theorem!Hilbert basis} Every submodule of that module is finitely generated.
\end{theorem}
\begin{theorem}[Henri\SubIndex{Cartan!Henri} Cartan \cite{Cartan:1944} p. 194 corollaire premiere, \cite{Grauert/Remmert:1984} p. 46, \cite{Gunning/Rossi:2009} p. 85 theorem 3, \cite{Taylor:2002} p. 274 theorem 11.2.6]\label{theorem:germ.convergence}
Every submodule of that module is closed under convergence.
\end{theorem}
\begin{proof}
Pick a submodule \(S\).
Take a convergent sequence \(f_i\to f\) of map germs, with \(f_i\in S\).
We need to prove that \(f\in S\).
Let \(R\) be the the ring of analytic function germs at the origin, \(I\subset R\) the germs vanishing at the origin, \(T\) the \(R\)-module of analytic map germs, which we identify with their Taylor series.
The \(k\)-th order Taylor series are the elements of \(T/I^{k+1}T\), and form a vector space of finite dimension, containing a linear subspace \(S/I^{k+1}S\), which is therefore a closed subset, so contains the \(k\)-th order Taylor coefficient of \(f\).
Let \(M\defeq T/S\), so \(f\in I^kM\) for all \(k=1,2,\dots\).
The set \(1+I\) is the set of germs of functions equal to \(1\) at the origin, so units in \(R\).
By Krull's intersection theorem, \(f=0\) in \(M\), i.e. \(f\in S\).
\end{proof}

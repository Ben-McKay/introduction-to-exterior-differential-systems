\chapter*{Preface}
To the reader who wants to dip a toe in the water: read chapter~\ref{chapter:eds}.
The reader who continues on to chapters~\ref{chapter:tableaux} and \ref{chapter:prolongation} will pick up the rest of the tools.
Subsequent chapters prove the theorems.
We assume that the reader is familiar with elementary differential geometry on manifolds and with differential forms.
These lectures explain how to apply the Cartan--K\"ahler theorem to problems in differential geometry.
Given some differential equations, we want to decide if they are locally solvable.
The Cartan--K\"ahler theorem gives a linear algebra test: if the equations pass the test, they are locally solvable.
We give the necessary background on partial differential equations in appendices~\ref{chapter:Cauchy.Kovalevskaya}, \ref{chapter:characteristics}, and the (not so necessary) background on moving frames in appendices~\ref{chapter:moving.frame}, \ref{chapter:Gauss.Bonnet}, \ref{chapter:geodesics}.
The reader should be aware of \cite{Cartan:1945}, which we will follow closely, and also the canonical reference work \cite{BCGGG:1991} and the canonical textbook \cite{Ivey/Landsberg:2003}.

I wrote these notes for lectures at the Banach Center in Warsaw, at the University of Paris Sud, and at the University of Rome Tor Vergata.
I thank Jan Gutt, Gianni Manno, Giovanni Moreno, Nefton Pali, and Filippo Bracci for invitations to give those lectures, and Francesco Russo for the opportunity to write these notes at the University of Catania.
 
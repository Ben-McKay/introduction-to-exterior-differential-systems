\chapter{Example: almost complex structures}
\chapterSummary{This chapter can be omitted without loss of continuity. 
As an application of the Cartan--K\"ahler theorem, we demonstrate the integrability of torsion free almost complex structures.}

\optionalSection{Example: the Cauchy--Riemann equations}%
Take several complex variables \(z^1,\dots,z^n\), with real and imaginary parts \(z^{\mu}=x^{\mu}+iy^{\mu}\).
A \emph{holomorphic}\define{holomorphic} function is a complex valued function \(f(z^1,\dots,z^n)\) satisfying the \emph{Cauchy--Riemann equations} \define{Cauchy--Riemann equations}\cite{Demailly:2012,Gunning/Rossi:2009,Hoermander:1990,Shabat:1992,Taylor:2002}\define{Cauchy--Riemann equations}
\[
\pderiv{f}{x^{\mu}}=i\pderiv{f}{y^{\mu}}, \quad \mu=1,2,\dots,n.
\]
In real and imaginary parts \(f=u+iv\), this becomes
\begin{align*}
\pderiv{u}{x^{\mu}}&=\pderiv{v}{y^{\mu}},\\
\pderiv{v}{x^{\mu}}&=-\pderiv{u}{y^{\mu}}.
\end{align*}
Let \(M\defeq\R[4n+2]_{x^{\mu},y^{\mu},u,v,p_{\mu},q_{\mu}}\) on which we take the exterior differential system \(\II\) generated by
\begin{align*}
\theta^1&\defeq du-p_{\mu}dx^{\mu}-q_{\mu}dy^{\mu},\\
\theta^2&\defeq dv+q_{\mu}dx^{\mu}-p_{\mu}dy^{\mu}.
\end{align*}
\[
d
\begin{pmatrix}
\theta^1\\
\theta^2
\end{pmatrix}
=
-
-\Tablo{*dp_1,*dp_2,.,*dp_n,dq_1,.,dq_n;
*-dq_1,*-dq_2,.,*-dq_n,dp_1,.,dp_n}
(1,2,.,n,n+1,.,2n)
[2,2,.,2,0,.,0]
\wedge
\begin{pmatrix}
dx^1\\
dx^2\\
\vdots\\
dx^n\\
dy^1\\
\vdots\\
dy^n
\end{pmatrix}
\]
So 
\[
\dim M+s_1+2s_2+\dots+2ns_{2n}=4n+2+n(n+1).
\]
Integral elements at each point \((x,y,u,v,p,q)\) are 
\[
\begin{pmatrix}
dp_{\mu}\\
dq_{\mu}
\end{pmatrix}
=
\begin{pmatrix}
r_{\mu\nu} & -s_{\mu\nu}\\
s_{\mu\nu} & r_{\mu\nu}
\end{pmatrix}
\begin{pmatrix}
dx^{\nu}\\
dy^{\nu}
\end{pmatrix}
\]
with \(r,s\) symmetric in lower indices.
So the space of integral elements has dimension \(4n+2+n(n+1)\), involution with general solution depending on \(2\) functions of \(n\) variables.

For a complex notation, let \(\omega^{\mu}\defeq dx^{\mu}+i\, dy^{\mu}\), \(\theta\defeq \theta^1+i\theta^2\), \(\pi_{\mu}\defeq dp_{\mu}+i\, dq_{\mu}\) and \(\omega^{\bar\mu}\defeq \overline{\omega^{\mu}}=dx^{\mu}-i\,dy^{\mu}\).
A tableau in complex differential forms will have terms expressed in wedge products with \(\omega^{\mu},\omega^{\bar{\mu}}\).
But for the Cauchy--Riemann equations we find
\[
d\theta=-%
\begin{pmatrix}
\freeDeriv{\pi^1} & 
\freeDeriv{\pi^2} & 
\dots &
\freeDeriv{\pi^n}
\end{pmatrix} 
\wedge
\begin{pmatrix}
\omega^1\\
\omega^2\\
\vdots\\
\omega^n
\end{pmatrix}
\]
with no \(\omega^{\bar{\mu}}\) terms.
The characters count complex polars, so double them to count real and imaginary parts: \(s_0,s_1,s_2,\dots,s_n=2,2,\dots,2\).
Integral elements are \(\pi_{\mu}=p_{\mu\nu}\omega^{\nu}\), with complex numbers \(p_{\mu\nu}=p_{\nu\mu}\), so \(n(n+1)/2\) complex numbers, hence \(n(n+1)\) real and imaginary parts.

\optionalSection{Example: almost complex structures}%
We can turn a real vector space into a complex vector space, just by picking a real linear map \(J\) so that \(J^2=-I\).
An \emph{almost complex structure}\define{almost complex structure} on an even dimensional manifold \(M\) is a choice of complex vector space structure on each tangent space, analytically varying, in that is described by an analytic map \(J\colon TM \to TM\), acting linearly on each tangent space, with \(J^2=-I\).

Complex Euclidean space has its usual almost complex structure \(J(x,y)=(-y,x)\), preserved by biholomorphisms (i.e. holomorphic diffeomorphisms) between open sets, as their derivatives are complex linear maps.
Any complex manifold \(M\) has an almost complex structure, defined by using holomorphic coordinates to identify locally with complex Euclidean space.

On an even dimensional manifold, a \emph{complex coframing} is a collection of complex valued \(1\)-forms \(\omega^{\mu}\) so that their real and imaginary parts coframe.
It is \emph{complex linear} for an almost complex structure if each \(\omega^{\mu}\) is a complex linear map on each tangent space, i.e. \(\omega^{\mu}\circ J = \sqrt{-1}\omega^{\mu}\) for every \(\mu\).
\prob{almost.complex:coframe.it}{Prove that every almost complex structure has, near each point, some complex linear complex coframing.}
\prob{almost.complex:coframe.ti}{Prove that every complex coframing is complex linear for a unique almost complex structure.}
Complex coframings are a useful way to exhibit almost complex structures.
\begin{example}
The complex coframing
\begin{align*}
\omega^1&\defeq dz,\\
\omega^2&\defeq dw-\bar{w} \, d\bar{z}
\end{align*}
yields a unique almost complex structure on the space parameterized by two complex variables \(z,w\).
\end{example}
\prob{almost.complex:coframe.ti.2}{Prove that any two complex coframings \(\omega^{\mu},\otomega^{\mu}\) yield the same almost complex structure just when \(\otomega^{\mu}=g^{\mu}_{\nu}\omega^{\nu}\) for a unique matrix \(g=(g^{\mu}_{\nu})\) of functions.}

Any complex differential form is expressed in any complex coframing \(\omega^{\mu}\) as a sum of wedge products of \(\omega^{\mu}\) and \(\bar\omega^{\mu}\). Following convention, write \(\bar\omega^{\mu}\) as \(\omega^{\bar{\mu}}\).
A \((p,q)\)-\emph{form} is a differential form expressed with \(p\) factors of \(\omega^{\mu}\) and \(q\) of \(\omega^{\bar{\mu}}\).
For example, \(\omega^{\mu}\) is \((1,0)\) while \(\omega^{\bar{\mu}}\) is \((0,1)\).
In particular,
\[
d\omega^{\mu}=%
t^{\mu}_{\mu\sigma}\omega^{\nu}\wedge\omega^{\sigma}+
t^{\mu}_{\nu\bar\sigma}\omega^{\nu}\wedge\omega^{\bar\sigma}+
t^{\mu}_{\bar\nu\sigma}\omega^{\bar\nu}\wedge\omega^{\sigma}+
t^{\mu}_{\bar\nu\bar\sigma}\omega^{\bar\nu}\wedge\omega^{\bar\sigma},
\]
for unique complex valued functions \(t\), antisymmetric in lower indices.
The exterior derivative of any \((p,q)\)-form is uniquely expressed as a sum of forms with \((p,q)\) raised by \((2,-1),(1,0),(0,1)\) or \((-1,2)\).
We thus split up
\[
d=\tau+\partial+\bar\partial+\bar\tau:
\]
\begin{alignat*}{3} 
\tau\omega^{\mu}
&=0,
\quad&&
\partial\omega^{\mu}
=t^{\mu}_{\nu\sigma}\omega^{\nu}\wedge\omega^{\sigma},\\
\bar\tau\omega^{\mu}
&=t^{\mu}_{\bar\nu\bar\sigma}\omega^{\bar\nu}\wedge\omega^{\bar\sigma},
\quad&&
\bar\partial\omega^{\mu}
=t^{\mu}_{\nu\bar\sigma}\omega^{\nu}\wedge\omega^{\bar\sigma}+
t^{\mu}_{\bar\sigma\nu}\omega^{\bar\sigma}\wedge\omega^{\nu}.
\end{alignat*}
Let \(\omega\defeq(\omega^{\mu})\).
Expanding out in the coframing, we find that \(0=\tau=\bar\tau\) on all differential forms if and only if \(\bar\tau\omega=0\).
Change coframing:  replace \(\omega\) by some \(g\omega\).
\[
d(g\omega)=dg\wedge\omega+gd\omega,
\]
expands out to
\begin{alignat*} {4}
\tau(g\omega)&=0,\qquad&&\partial(g\omega)&=(\partial g)\wedge\omega+g \, \partial\omega,\\
\bar\tau(g\omega)&=g\,\bar\tau\omega,\qquad&&\bar\partial(g\omega)&=(\bar\partial g)\wedge\omega+g \, \bar\partial\omega.
\end{alignat*}
Hence \(\bar\tau\omega=0\) just when \(\bar\tau(g\omega)=0\).
So vanishing of \(\bar\tau\) is a property of the almost complex structure.
\begin{problem}{almost.complex:dual}
Prove that \(\bar\tau=0\) just when \(\tau=0\).
\end{problem}
\begin{problem}{almost.complex:Nijenhuis}
Construct a tensor whose vanishing is equivalent to \(\bar\tau=0\).
\end{problem}
\begin{answer}{almost.complex:Nijenhuis}
If \(u_{\mu},v_{\mu}\) are the vector fields dual to the real and imaginary parts of the \(\omega^{\mu}\), the \emph{torsion tensor}\define{torsion!tensor}, or \emph{Nijenhuis tensor}\define{Nijenhuis tensor}
\[
T\defeq
(u_{\mu}+iv_{\mu})\tau\omega^{\mu}
\]
depends only on the almost complex structure, a section of \(\nForms{0,2}{M} \otimes TM \otimes \C\).
\end{answer}
\begin{example}
The complex coframing
\begin{align*}
\omega^1&\defeq dz,\\
\omega^2&\defeq dw-|w|^2 \, d\bar{z}
\end{align*}
has
\begin{align*}
\bar\tau\omega^1&=0,\\
\bar\tau\omega^2&=-w\, d\bar{w} \wedge d\bar{z},\\
&=w\,\omega^{\bar{1}}\wedge\omega^{\bar{2}},\\
&\ne0.
\end{align*}
Therefore the associated almost complex structure is not a complex structure.
\end{example}
A complex valued function \(f\colon M \to\C\) on an almost complex manifold is \emph{holomorphic} if \(df\) is complex linear.
\begin{lemma}
An almost complex manifold which admits holomorphic functions with arbitrary complex linear differentials at each point has \(\bar\tau=0\).
\end{lemma}
\begin{proof}
Denote by \(2n\) the dimension of \(M\) as a real manifold.
Since this problem is local, we can assume that \(M\) has a global complex linear coframing \(\omega^{\mu}\).
Take the manifold \(M'\defeq M \times \C_z \times \C[n]_{Z}\), and the exterior differential system generated by 
\(
dz-Z_{\mu}\omega^{\mu}.
\)
Any integral manifold on which \(\omega^{\mu}\) are complex-linearly independent is locally the graph of a holomorphic function, and vice versa.
The tableau
\begin{align*}
d(dz-Z\omega)
&=-dZ\wedge\omega-Z\,d\omega,
\\
&=
-DZ\wedge\omega-Z\, \bar\tau\omega,
\end{align*}
where
\[
DZ_{\mu}\defeq dZ_{\mu}+Z_{\sigma}(t^{\sigma}_{\nu\mu}\omega^{\nu}+2t^{\sigma}_{\bar{\nu}\mu}\omega^{\bar{\nu}}).
\]
The torsion, where \(Z\ne0\), consists of the expression \(Z\,\bar\tau\omega=Z_{\mu}\bar\tau\omega^{\mu}\).
\end{proof}
\begin{problem}{almost.complex:hol.fn}
Find all holomorphic functions for the almost complex structure of the complex coframing
\begin{align*}
\omega^1&\defeq dz,\\
\omega^2&\defeq dw-\bar{w} \, d\bar{z}
\end{align*}
\end{problem}
Any wedge product \(\pi\wedge\omega\) of complex valued \(1\)-forms can be rewritten as a real wedge product in real and imaginary parts
\[
\begin{pmatrix}
\pi^1 & -\pi^2\\
\pi^2 & \pi^1
\end{pmatrix}
\wedge
\begin{pmatrix}
\omega^1 & -\omega^2\\
\omega^2 & \omega^1
\end{pmatrix}.
\]
If \(\pi\wedge\omega\) occurs in a tableau, at some grade,  it contributes \(2\) linearly independent \(1\)-forms:
\[
\begin{pmatrix}
\freeDeriv{\pi^1} & -\pi^2\\
\freeDeriv{\pi^2} & \pi^1
\end{pmatrix}
\wedge
\begin{pmatrix}
\omega^1 & -\omega^2\\
\omega^2 & \omega^1
\end{pmatrix}.
\]
Count with a complex tableau as if it were real linear, but double the characters.
\begin{theorem}\label{theorem:analytic.NN}
An analytic almost complex structure has \(\bar\tau=0\) just when it arises from a complex structure.
\end{theorem}
This theorem remains true with milder assumptions than analyticity \cite{Hill/Taylor:2003}.
\begin{proof}
On a complex manifold with holomorphic coordinates \(z^{\mu}\), the \(1\)-forms \(\omega^{\mu}=dz^{\mu}\) are complex linear for the standard almost complex structure, and have \(d\omega^{\mu}=0\), so no torsion.

Take an almost complex structure \(J\) which has \(\bar\tau=0\).
Our problem is to construct local holomorphic coordinate functions locally identifying \(J\) with the standard complex structure on complex Euclidean space.
Again take the exterior differential system generated by \(\theta\defeq dz-Z\omega\):
\[
d\theta=-%
\Tablo{*DZ_1,*DZ_2,.,*DZ_n}(1,2,.,n)[2,2,.,2]
\wedge
\begin{pmatrix}
\omega^1\\
\omega^2\\
\vdots\\
\omega^n
\end{pmatrix}.
\]
Integral elements are \(DZ_{\mu}=p_{\mu\nu}\omega^{\nu}\), \(p_{\mu\nu}=p_{\nu\mu}\), \(n(n+1)/2\) complex constants, so \(n(n+1)\) real constants, involution.
So there are holomorphic functions with arbitrary differentials at a point, i.e. local holomorphic coordinates.
\end{proof}
\begin{example}
The complex coframing
\begin{align*}
\omega^1&=dz,\\
\omega^2&=dw-w\,d\bar{z},
\end{align*}
determines a complex structure.
\end{example}
\begin{example}
The expression \(\bar\tau\omega^{\mu}=t^{\mu}_{\bar\nu\bar\sigma}\omega^{\bar{\nu}}\wedge\omega^{\bar{\sigma}}\) is antisymmetric in \(\bar\nu,\bar\sigma\).
It vanishes if \(M\) has complex dimension \(1\), i.e. real dimension \(2\):
every almost complex manifold of real dimension \(2\) is a Riemann surface.
\end{example}
\begin{example}
In this example, we assume familiarity with matrix groups \cite{Stillwell:2008}.
The manifold \(\SU{3}\) is the collection of all \(3\times3\) complex unitary matrices \(z=(z_{\mu\bar\nu})\) of determinant \(1\).
Write \(z_{\bar\mu\nu}\) to mean \(\bar{z}_{\mu\bar\nu}\), so that \(z^*\) has entries \(z^*_{\mu\bar\nu}=z_{\bar\nu\mu}\).
Unitarity is \(z_{\mu\bar\sigma}z_{\bar\nu\sigma}=\delta_{\mu\bar\nu}\).
Note that \(\SU{3}\) is a real submanifold, \emph{not} a complex submanifold, of the \(3\times3\) matrices, as this unitarity equation is not complex analytic.

Since \(\SU{3}\) is a submanifold of matrices, each tangent vector \(v\) to \(\SU{3}\) is expressed as a matrix.
The tangent space \(T_I \SU{3}\) at the identity matrix, denoted \(\LieSU_3\), is the set of all traceless skew adjoint \(3\times3\) complex matrices.

It is traditional to write the identity function on any group as \(g\), so \(g(z)=z\).
The \emph{Maurer--Cartan}\define{Maurer--Cartan form} \(1\)-form \(\omega\defeq g^{-1}dg\) is a \(1\)-form on \(\SU{3}\), valued in \(\LieSU_3\), i.e. to each tangent vector \(v\in T_z \SU{3}\), which we identify with a matrix \(A\), \(v\hook\omega=z^{-1}A\).
Write out \(\omega\) as a matrix of complex valued \(1\)-forms
\[
\omega
=
\begin{pmatrix}
\omega_{1\bar{1}}&\omega_{1\bar{2}}&\omega_{1\bar{3}}\\
\omega_{2\bar{1}}&\omega_{2\bar{2}}&\omega_{2\bar{3}}\\
\omega_{3\bar{1}}&\omega_{3\bar{2}}&\omega_{3\bar{3}}
\end{pmatrix}
\]
with \(\omega_{\mu\bar\nu}=-\omega_{\bar\nu\mu}\).
\prob{almost.complex:SU.left}{Prove that \(\omega\) is invariant under left translation.}
\begin{answer}{almost.complex:SU.left}
Consider left action of \(\SU{3}\) on itself: \(L_h z=hz\).
The identity function \(g(z)=z\) behaves like \((L_h^*g)(z)=g(L_hz)=g(hz)=hz=hg(z)\), so \(L_h^*g=hg\).
Thus for any constant matrix \(h\in\SU{3}\),
\begin{align*}
L_h^*\omega
&=L_h^*(g^{-1}dg),
\\
&=(L_h^*g)^{-1}dL_h^*g,
\\
&=
(hg)^{-1}d(hg),
\\
&=
g^{-1}h^{-1}h \, dg,
\\
&=
g^{-1}\,dg,
\\
&=\omega.
\end{align*}
\end{answer}
\prob{almost.complex:dSU}{Calculate that \(d\omega=-\omega\wedge\omega\).}
\begin{answer}{almost.complex:dSU}
Differentiating \(\omega=g^{-1}\,dg\), i.e. \(dg=g\omega\),
\begin{align*}
0&=
dg\wedge\omega+g\,d\omega,
\\
&=
g\omega\wedge\omega+g\,d\omega,
\end{align*}
we find 
\[
d\omega=-\omega\wedge\omega.
\]
\end{answer}
In matrix entries,
\(d\omega_{\mu\bar\nu}=-\omega_{\mu\bar\sigma}\wedge\omega_{\sigma\bar\nu}\).
Consider the coframing 
\[
\omega_{1\bar{1}}+i\omega_{2\bar{2}},\omega_{1\bar{2}},\omega_{1\bar{3}},\omega_{2\bar{3}},
\]
\prob{almost.complex:torone}{Take exterior derivatives and find torsion vanishing: \(\SU{3}\) has a left invariant complex structure.}
\prob{almost:complex:tortwo}{On the other hand, if we conjugate one of the last three \(1\)-forms in the coframing, prove that we arrive at a left invariant almost complex structure which is not complex.}
\end{example}
\begin{problem*}{almost.complex:Weil}Prove theorem~\vref{theorem:analytic.NN} by complexifying variables locally, and applying the Frobenius theorem.\end{problem*}
\begin{answer}{almost.complex:Weil}
\cite{Weil:1958} pp. 36--37
\end{answer}
\optionalSection{Almost complex submanifolds}%
An \emph{almost complex submanifold} of an almost complex manifold \(M\) is a submanifold whose tangent planes are complex linear subspaces.
Suppose that \(M\) has real dimension \(2(p+q)\).
Let's look for almost complex submanifolds of dimension \(2p\).
Take a complex linear coframing \(\omega^1,\dots,\omega^p,\pi^1,\dots,\pi^q\).
Take an almost complex submanifold of real dimension \(2p\) on which \(\omega^1,\dots,\omega^p\) have linearly independent real and imaginary parts.
Then on that submanifold, \(\pi=p\omega\) for a complex matrix \(p\).
Our almost complex submanifold is an integral manifold of the exterior differential system \(\pi=p\omega\) on \(M\times\C[pq]\).
We let \(\theta\defeq\pi-p\omega\).
Then on any integral manifold \(0=\bar\tau\theta=\bar\tau\pi-p \, \bar\tau\omega\).
\begin{problem}{almost.complex:sub}
Prove that, for any \(p>1\), if there is an almost complex submanifold of real dimension \(2p\) tangent to any complex linear subspace of complex dimension \(p\) in any tangent space, then \(M\) is a complex manifold.
\end{problem}
A \emph{holomorphic curve}\define{holomorphic!curve}, often called a \emph{pseudoholomorphic curve}, in an almost complex manifold \(M\) is a map \(C \to M\) from a Riemann surface, with complex linear differential.
\begin{problem}{almost.complex:curve}
Prove the existence of embedded holomorphic disks, i.e. embedded holomorphic curves \(C \to M\) where \(C\) is the unit disk in the complex plane, in every analytic almost complex manifold, tangent to every complex line in every tangent space.
\end{problem}
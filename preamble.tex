\documentclass{willowtreebook}
\Title{Introduction \\ to Exterior \\ Differential \\ Systems}
\Author{\texorpdfstring{Benjamin \scotsMc{}Kay}{Benjamin McKay}}
\BibliographyFile{main}
\usepackage[french,english]{babel}
\usepackage{morewrites}
\usepackage{standalone}
\usepackage{enumitem}
\usepackage{xspace}
\usepackage{amsfonts}
\usepackage{amssymb}
\usepackage{mathtools}
\usepackage{braket}
\usepackage{varioref}
\usepackage{xstring}
\usepackage{xparse}
\usepackage{array}
\usepackage{blkarray}
\usepackage{longtable}
\usepackage{multirow}
\usepackage[math]{cellspace}
\usepackage{changepage}
\usepackage{epigraph-keys}
\usepackage{tikz}
\usetikzlibrary{fadings,calc,hobby, arrows.meta,bending,decorations.markings,babel}
\usepackage{tikz-cd}
\usepackage{pgfplots}\pgfplotsset{compat=1.16}
\usepackage[customcolors]{hf-tikz}
\NewDocumentCommand\textprime{}{\ensuremath{'}}
\NewDocumentCommand\pderiv{omm}%
{%
\IfValueTF{#1}%
{%%
\frac{\partial^{#1} #2}{\partial #3^{#1}}%
}%%
{%%
\frac{\partial #2}{\partial #3}%
}%%
}%
\tikzfading[name=fade out, inner color=transparent!0, outer color=transparent!100]
% We don't need to worry about the pdf page groups, so we ignore these warnings.
\pdfsuppresswarningpagegroup=1
\NewDocumentCommand\transpose{m}%
{%
	#1^{\!\top}%
}%

%%% THIS IS MY CURRENT APPROACH TO WRITING TABLEAU.
% You have to write . to mean \dots.
\tikzset{free/.style={fill=blue!30,draw=gray!16,thick,rounded corners}}
\newlength\tableauEntryLength%
\newlength\nextTableauEntryLength%
\newcount\tableauColumns%
\newcount\tableauRows%
\newcount\soFarColumn%
\newcount\soFarGrade%
\newcount\numberOfGradingColumns%
\newcount\tableauDotPosition%
\newcommand\tableauEntryPadding{.7em}%
\NewDocumentCommand\Tablo{md()o}%
{%
	\edef\tableauContents{#1}%
	\StrCount{\tableauContents}{;}[\tableauRowsString]%
	\global\tableauRows\tableauRowsString\relax%
	\global\advance\tableauRows by 1\relax%
	\StrBefore{\tableauContents;}{;}[\firstTableauRow]%
	\tableauColumns0\relax%
	\StrCount{\firstTableauRow}{,}[\tableauColumnsString]%
	\global\tableauColumns\tableauColumnsString\relax%
	\global\advance\tableauColumns by 1\relax%
	\StrSubstitute{\tableauContents}{;}{,}[\tableauWithoutSemicolons]%
	\StrSubstitute{\tableauWithoutSemicolons}%
		{*}{|[free]|}[\tableauWithoutSemicolons]%
	\setlength\tableauEntryLength{0pt}%
	\global\tableauEntryLength=\tableauEntryLength\relax%
	\let\mymatrixcontent\empty%
	\let\gradingColumns\empty%
	\numberOfGradingColumns0\relax%
	\global\soFarColumn0\relax%
  	\foreach \c in \tableauWithoutSemicolons{%
		\StrPosition{\c}{!}[\exclamationPosition]%
		\StrSubstitute{\c}{!}{}[\cNoExclamation]%
		\StrPosition{\c}{.}[\theDotPosition]%
		\tableauDotPosition\theDotPosition\relax%				
		\StrSubstitute{\cNoExclamation}{.}{}[\cNoExclamation]%
		\ifnum\tableauDotPosition=0\relax%
			\expandafter\gappto\expandafter\mymatrixcontent\expandafter{\cNoExclamation }%
		\else%
			\expandafter\gappto\expandafter\mymatrixcontent\expandafter{\cNoExclamation }%
			\gappto\mymatrixcontent{\dots}%
		\fi%
		\global\advance\soFarColumn by 1\relax%
		\ifnum0=\exclamationPosition%
		\else%
			\global\advance\numberOfGradingColumns by 1\relax%
			\edef\soFarColumnText{\the\soFarColumn}%
			\ifnum\numberOfGradingColumns>1\relax%
				\expandafter\gappto\expandafter\gradingColumns\expandafter{,}%
			\fi%
			\expandafter\gappto\expandafter\gradingColumns\expandafter{\soFarColumnText}%
		\fi%
		\ifnum\soFarColumn=\the\tableauColumns%
			\expandafter\gappto\expandafter\mymatrixcontent\expandafter{ \\ }%
			\global\soFarColumn0\relax%
		\else%
			\expandafter\gappto\expandafter\mymatrixcontent\expandafter{ \& }%	
		\fi%
		\StrSubstitute{\cNoExclamation}{|[free]|}{}[\cMathEntry]%
		\settowidth{\nextTableauEntryLength}{\({\cMathEntry}\)}%
		\setlength{\tableauEntryLength}%
			{\maxof%
				{\tableauEntryLength}%
				{\nextTableauEntryLength}%
			}%
		\global\tableauEntryLength=\tableauEntryLength\relax%
    }%
	\addtolength{\tableauEntryLength}{\tableauEntryPadding}%
	\begin{tikzpicture}[baseline=0]%
	\matrix (m) [%
		matrix of math nodes,%
			nodes={%
    	    	rectangle,%
				minimum width=\tableauEntryLength,%
				minimum height=1.5em,%
				text depth=0.25ex,%
				inner sep=0pt, 
				outer sep=0pt,%
			},%
		nodes in empty cells,
		inner sep=0pt,%
		left delimiter=(,%
		right delimiter=),%
		ampersand replacement=\&]%
		{%%
			\mymatrixcontent%
		};%%
		\node (a) at (m-\the\tableauRows-1){};
		\node (b) at (m-\the\tableauRows-\the\tableauColumns){};
		\node (mbot) at ($(a)!0.5!(b)$) {};
		\foreach \i in \gradingColumns%
		{%
			\draw (m-1-\i.north west) -- (m-\the\tableauRows-\i.south west);
		}%
		\IfValueTF{#2}%
		{%
			\edef\tableauCharacters{#3}%
			\let\mymatrixcontent\empty%
			\soFarColumn0\relax%
			\soFarGrade0\relax%
	 		\foreach \c in \tableauCharacters%
			{%
				\global\advance\soFarColumn by 1\relax%
				\StrPosition{\c}{.}[\theDotPosition]%
				\tableauDotPosition\theDotPosition\relax%				
				\ifnum\tableauDotPosition=0\relax%
					\settowidth{\nextTableauEntryLength}{\({\c}\)}%
					\ifdim\nextTableauEntryLength>1pt%
						\global\advance\soFarGrade by 1\relax%
						\expandafter\gappto\expandafter\mymatrixcontent\expandafter{\c }%
					\fi%
				\else%
					\gappto\mymatrixcontent{\dots}%
				\fi%
				\ifnum\soFarColumn=\the\tableauColumns%
					\expandafter\gappto\expandafter\mymatrixcontent\expandafter{ \\ }%
				\else%
					\expandafter\gappto\expandafter\mymatrixcontent\expandafter{ \& }%	
				\fi%
			}%
			\edef\tableauCharacterLabels{#2}%
			\global\let\mylabelmatrixcontent\empty%
			\soFarColumn0\relax%
			\soFarGrade0\relax%
	 		\foreach \c in \tableauCharacterLabels%
			{%
				\global\advance\soFarColumn by 1\relax%
				\StrPosition{\c}{.}[\theDotPosition]%
				\tableauDotPosition\theDotPosition\relax%				
				\ifnum\tableauDotPosition=0\relax%
					\settowidth{\nextTableauEntryLength}{\({\c}\)}%
					\ifdim\nextTableauEntryLength>1pt%
						\global\advance\soFarGrade by 1\relax%
						\edef\colLabel{s_{\c}}%
						\expandafter\gappto\expandafter\mylabelmatrixcontent\expandafter{\colLabel }%
					\fi%
				\else%
					\gappto\mylabelmatrixcontent{\dots}%
				\fi%
				\ifnum\soFarColumn=\the\tableauColumns%
					\expandafter\gappto\expandafter\mylabelmatrixcontent\expandafter{ \\ }%
				\else%
					\expandafter\gappto\expandafter\mylabelmatrixcontent\expandafter{ \& }%	
				\fi%
			}%
		}%
		{%
			\IfValueT{#3}%
			{%
				\edef\tableauCharacters{#3}%
				\let\mymatrixcontent\empty%
				\let\mylabelmatrixcontent\empty%
				\soFarColumn0\relax%
				\soFarGrade0\relax%
		  		\foreach \c in \tableauCharacters%
		  		{%
					\global\advance\soFarColumn by 1\relax%
					\settowidth{\nextTableauEntryLength}{\({\c}\)}%
					\ifdim\nextTableauEntryLength>1pt%
						\global\advance\soFarGrade by 1\relax%
						\edef\colLabel{s_{\the\soFarGrade}}%
						\expandafter\gappto\expandafter\mylabelmatrixcontent\expandafter{\colLabel }%
						\expandafter\gappto\expandafter\mymatrixcontent\expandafter{\c }%
					\fi%
					\ifnum\soFarColumn=\the\tableauColumns%
						\expandafter\gappto\expandafter\mylabelmatrixcontent\expandafter{ \\ }%
						\expandafter\gappto\expandafter\mymatrixcontent\expandafter{ \\ }%
					\else%
						\expandafter\gappto\expandafter\mylabelmatrixcontent\expandafter{ \& }%	
						\expandafter\gappto\expandafter\mymatrixcontent\expandafter{ \& }%	
					\fi%
				}%
			}%
		}%
		\IfValueT{#3}%
		{%
			\matrix (n) [%
			below of=mbot,%
			matrix of math nodes,%
				nodes={%
   		     		rectangle,%
					minimum width=\tableauEntryLength,%
					minimum height=1.5em,%
					text depth=0.25ex,%
					inner sep=0pt, 
					outer sep=0pt,%
					},%
			nodes in empty cells,
			inner sep=0pt,%
			ampersand replacement=\&]%
			{%%
				\mylabelmatrixcontent%
				\mymatrixcontent%
			};%%
		}%
	\end{tikzpicture}%
}%

%%% THIS IS MY OLD APPROACH TO WRITING TABLEAU:
\NewDocumentEnvironment{gradedTableau}{m}%%
{%%
\arrayrulecolor{black}%
\rulecolor{black}%
\def\arraystretch{1.2}%
\left(\begin{array}{#1}%
}%%
{%%
\end{array}\right)%
\arrayrulecolor{gray!30}%
\rulecolor{gray!30}%
}%%

\NewDocumentCommand\+{}{\\ \hline}
\let\emptyGrade=\hline

\NewDocumentEnvironment{gradedIndependents}{}%%
{%%
\begin{gradedTableau}{c}
}%%
{%%
\end{gradedTableau}
}%%

\NewDocumentEnvironment{tableau}{}%%
{%%
\left( \ \begin{matrix}
}%%
{%%
\end{matrix} \ \right)
}%%
\NewDocumentCommand\freeDeriv{om}%
{%
	\settowidth{\tableauEntryLength}{\({#2}\)}%
	\addtolength{\tableauEntryLength}{\tableauEntryPadding}%
	\begin{tikzpicture}[baseline=\IfValueTF{#1}{#1}{-.5em}]%
	\matrix (m) [%
		matrix of math nodes,%
			nodes={%
    	    	rectangle,%
				minimum width=\tableauEntryLength,%
				minimum height=1.5em,%
				text depth=0.25ex,%
				inner sep=0pt, 
				outer sep=0pt,%
			},%
		inner sep=0pt]%
		{|[free]|#2\\};%
	\end{tikzpicture}%
}%


%% Commands
\newcommand*{\pr}[1]{\ensuremath{\left(#1\right)}}
\newcommand*{\of}[1]{\ensuremath{\!\pr{#1}}}
\NewDocumentCommand\bbX{mo}{\ensuremath{\IfValueTF{#2}{\mathbb{#1}^{#2}}{\mathbb{#1}}}}
\NewDocumentCommand\MakeBB{m}%
{%
	\expandafter\def\csname #1\endcsname{\bbX{#1}}%
}%
\def\lst{C,E,R,Z}
\makeatletter
\@for\i:=\lst\do{\expandafter\MakeBB \i}
\makeatother
\newcommand*{\GL}[1]{\ensuremath{\operatorname{GL}\of{#1}}}
\newcommand*{\SU}[1]{\ensuremath{\operatorname{SU}_{#1}}}
\newcommand*{\LieDer}{\ensuremath{\EuScript L}}
\newcommand*{\defeq}{\mathrel{\vcenter{\baselineskip0.5ex \lineskiplimit0pt
                     \hbox{\scriptsize.}\hbox{\scriptsize.}}}%
                     =}
\newcommand*{\p}[1]{\ensuremath{\partial_{#1}}}


\makeatletter
\DeclareFontFamily{OMX}{MnSymbolE}{}
\DeclareSymbolFont{MnLargeSymbols}{OMX}{MnSymbolE}{m}{n}
\SetSymbolFont{MnLargeSymbols}{bold}{OMX}{MnSymbolE}{b}{n}
\DeclareFontShape{OMX}{MnSymbolE}{m}{n}{
<-6>  MnSymbolE5
<6-7>  MnSymbolE6
<7-8>  MnSymbolE7
<8-9>  MnSymbolE8
<9-10> MnSymbolE9
<10-12> MnSymbolE10
<12->   MnSymbolE12
}{}
\DeclareFontShape{OMX}{MnSymbolE}{b}{n}{
<-6>  MnSymbolE-Bold5
<6-7>  MnSymbolE-Bold6
<7-8>  MnSymbolE-Bold7
<8-9>  MnSymbolE-Bold8
<9-10> MnSymbolE-Bold9
<10-12> MnSymbolE-Bold10
<12->   MnSymbolE-Bold12
}{}
\DeclareMathDelimiter{\ulcorner}{\mathopen}{MnLargeSymbols}{'036}{MnLargeSymbols}{'036}
\DeclareMathDelimiter{\urcorner}{\mathclose}{MnLargeSymbols}{'043}{MnLargeSymbols}{'043}
\makeatother

\newcommand*{\frameBundle}[1]{\ensuremath{\left\ulcorner\!#1\right.}}
\newcommand*{\frameBundleE}[1]{\ensuremath{\left\ulcorner\!\mathbb{E}^{#1}\right.}}
\newcommand*{\adaptedFrameBundle}[2]{\ensuremath{\left\ulcorner\!#2_{#1}\right.}}
\newcommand*{\op}[1]{\ensuremath{\operatorname{open} \subset #1}}
\NewDocumentCommand\LieG{}{\ensuremath{\mathfrak{g}}}
\NewDocumentCommand\LieSL{e_o}{\ensuremath{\mathfrak{sl}_{#1}\IfValueTF{#2}{#2}}}
\NewDocumentCommand\LieSU{e_o}{\ensuremath{\mathfrak{su}_{#1}\IfValueTF{#2}{#2}}}
\NewDocumentCommand\LieSO{m}{\ensuremath{\mathfrak{so}_{#1}}}
\NewDocumentCommand\Romanbar{m}{%
\relax\ifmmode%
\IfStrEqCase{#1}{
{1}{I}%
{2}{I\kern-2.5pt I}%
{3}{I\kern-2.5pt I\kern-2.5pt I}%
{4}{I\kern-1pt V}%
{5}{V}%
{6}{V\kern-2.6pt I}%
{7}{V\kern-2.6pt I\kern-2.5pt I}%
{8}{V\kern-2.6pt I\kern-2.5pt I\kern-2.5pt I}%
{9}{I\kern-2.5pt X}%
{10}{X}%
{11}{X\kern-2.8pt I}%
}%
\else%
\IfStrEqCase{#1}{
{1}{I}%
{2}{I\kern-0.57pt I}%
{3}{I\kern-0.57pt I\kern-0.57pt I}%
{4}{I\kern-1pt V}%
{5}{V}%
{6}{V\kern-.49pt I}%
{7}{V\kern-.49pt I\kern-0.57pt I}%
{8}{V\kern-.49pt I\kern-0.57pt I\kern-0.57pt I}%
{9}{I\kern-0.66pt X}%
{10}{X}%
{11}{X\kern-1pt I}%
}%
\fi%
}
\newcommand{\shapeOp}{\ensuremath{\Romanbar{2}}}
\newcommand{\thirdFundForm}{\ensuremath{\Romanbar{3}}}
\newcommand*{\II}{\ensuremath{\mathcal{I}}}
\newcommand*{\JJ}{\ensuremath{\mathcal{J}}}
\newcommand{\spn}[1]{\ensuremath{\left<#1\right>}}
\NewDocumentCommand\prl{mO{1}}{\ensuremath{#1^{(#2)}}}
\NewDocumentCommand\prob{mm}%
{%
\begin{problem}{#1}#2\end{problem}%
}%

% Hooking a vector into a differential form:
\newcommand{\hook}{\ensuremath{\mathbin{ \hbox{\vrule height1.4pt
        width4pt depth-1pt \vrule height4pt width0.4pt depth-1pt}}}}

\NewDocumentCommand\Lm{smm}{\IfBooleanTF{#1}{\ensuremath{\Lambda^{#2}\left(#3\right)}}{\ensuremath{\Lambda^{#2}#3}}}
\NewDocumentCommand\Lmtop{sm}{\IfBooleanTF{#1}{\ensuremath{\Lambda^{\operatorname{top}}\left(#2\right)}}{\ensuremath{\Lambda^{\operatorname{top}}#2}}}
\NewDocumentCommand\Gr{mm}{\ensuremath{\operatorname{Gr}_{#1}\!#2}}
\NewDocumentCommand\nForms{mm}{\ensuremath{\Omega^{#1}_{#2}}}

% Characteristic variety: \CV^n, \CV_k, \CV^n_k, \CV(E), \CV^n(E)
\NewDocumentCommand\CV{e^e_d()}{%
    \IfNoValueTF{#1}{\IfNoValueTF{#2}{\Xi}{\Xi_{#2}}}%
    {\IfNoValueTF{#2}{\Xi^{#1}}{\Xi^{#1}_{#2}}}%
    \IfValueTF{#3}{\!\left(#3\right)}{}%
}%

% Projective space: \Proj^n, \Proj_k, \Proj^n_k, \Proj(V), \Proj^n(k)
\NewDocumentCommand\Proj{e^e_d()}{%
    \IfNoValueTF{#1}{\IfNoValueTF{#2}{\mathbb{P}}{\mathbb{P}_{\!#2}}}%
    {\IfNoValueTF{#2}{\mathbb{P}^{#1}}{\mathbb{P}^{#1}_{\!#2}}}%
    \IfValueTF{#3}{#3}{}%
}%

%\NewDocumentCommand\mapsfrom{}{\mathrel{\reflectbox{\ensuremath{\mapsto}}}}

\DeclareMathOperator{\codim}{codim}
% little o
\NewDocumentCommand\littleo{m}{\ensuremath{o\of{#1}}}
\NewDocumentCommand\opsymbol{om}%
{%%
\IfValueTF{#1}{\sigma_{#1}\of{#2}}{\sigma_{#2}}%
}%%
\NewDocumentCommand\charvariety{om}%
{%%
\ensuremath{%%%
\IfValueTF{#1}{\Xi_{#1,#2}}{\Xi_{#2}}%
}%%%
}%%
\NewDocumentCommand\complexcharvariety{om}%
{%%
\ensuremath{\charvariety[#1]{#2}^{\mathbb{C}}}
}%%
\newcommand*{\Sym}[2]{\ensuremath{\operatorname{Sym}^{#1}\of{#2}}}
\DeclareMathOperator{\Kernel}{ker}

\NewDocumentCommand\pf{}{\omega}
\NewDocumentCommand\pc{}{\gamma}
\NewDocumentCommand\qf{}{\otomega}
\NewDocumentCommand\qc{}{\otgamma}
\NewDocumentCommand\ot{m}%
{%
#1'%
%\mathsf{#1}%
}%
\NewDocumentCommand\otalpha{}{\alpha'}
\NewDocumentCommand\otgamma{}{\gamma'}
\NewDocumentCommand\otpi{}{\pi'}
\NewDocumentCommand\otrho{}{\rho'}
\NewDocumentCommand\otomega{}{\omega'}
\NewDocumentCommand\Gp{mm}{\ensuremath{\textrm{#1}\of{#2}}}
\NewDocumentCommand\Orth{m}{\Gp{O}{#1}}
\NewDocumentCommand\SO{m}{\Gp{SO}{#1}}
\NewDocumentCommand{\norm}{moo}%
{%%
\IfValueTF{#2}%
{%
\IfValueTF{#3}%%%
{%%%
\left|#1\right|_{L^{#2}\of{#3}}
}%%%
{%%%
\left|#1\right|_{L^{#2}}
}%%%
}%
{%
\left|#1\right|
}%
}%%
\DeclareMathOperator{\indx}{index}
% Inner product 
\newcommand*{\ip}[2]{\ensuremath{\left<#1,#2\right>}}
\newcommand*{\geodesicVectorField}{\ensuremath{\mathfrak{X}}}
% length
\NewDocumentCommand\length{sm}%
{%
\IfBooleanTF{#1}%
{\ensuremath{\operatorname{length}\of{#2}}}%
{\ensuremath{\operatorname{length}#2}}
}%
\newcommand*{\id}{\ensuremath{\operatorname{id}}}

% Lie bracket
\NewDocumentCommand\lb{smm}{\IfBooleanTF{#1}{\left[{#2},{#3}\right]}{[{#2}{#3}]}}

% L^2 inner product
\NewDocumentCommand\LtwoIP{omm}%
{%
	\ensuremath{\left<#2,#3\right>_{L^2\IfValueT{#1}{\left(#1\right)}}}%
}%

\NewDocumentCommand\linearIn{m}{\text{linear in } #1}

\newcommand{\anyMatrix}{\vcenter{\hbox{\rule{.5ex}{.5ex}}}}

\newcommand\scalemath[2]{\scalebox{#1}{\mbox{\ensuremath{\displaystyle #2}}}}
\NewDocumentCommand\LessOrGreaterGraded{mm}%
{%
	\scalemath{.5}{#1}#2%
}%
\NewDocumentCommand\legr{m}%
{%
	\LessOrGreaterGraded{\le}{#1}%
}%
\NewDocumentCommand\ggr{m}%
{%
	\LessOrGreaterGraded{>}{#1}%
}%

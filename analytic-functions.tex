\chapter{Analytic functions}\label{chapter:analytic.functions}%
\chapterSummary{A review of the theory of Taylor series and analytic functions.}%
\section{Formal Taylor series}
For \(x_1,x_2,\dots,x_n\) real variables and \(a_1,a_2,\dots,a_n\) nonnegative integers, let
\begin{align*}
x&\defeq(x_1,x_2,\dots,x_n),\\
a&\defeq(a_1,a_2,\dots,a_n),\\ 
x^a&\defeq x_1^{a_1} \dots x_n^{a_n},\\
a!&\defeq a_1!\dots a_n!\text{ and}\\ 
\partial^a&\defeq\frac{\partial^{a_1}}{\partial x_1^{a_1}} \dots \frac{\partial^{a_n}}{\partial x_n^{a_n}}.
\end{align*}
A \emph{formal Taylor series}\define{Taylor series!formal}\define{formal!Taylor series} is an expression \(\sum c_a \pr{x-x_0}^a\) with real constants \(c_a\), not required to converge.
\begin{problem}{cauchy:formal}
Prove that the formal Taylor series \(\sum_n (2n)! t^n\) diverges for \(t \ne 0\).
\end{problem}
We add, subtract, multiply, differentiate and compose in the obvious way: finitely many terms at a time.
Crucially, each output term depends only on input terms of lower or equal order (or, for the derivative, on just one order higher), so on only \emph{finitely} many input terms.
When we add, multiply, differentiate or compose, each step is only adding or multiplying coefficients.
In particular, the sum, product, derivative (in any variable) and composition of formal Taylor series with positive terms has positive terms.

A formal Taylor series \(\sum b_a x^a\) \emph{majorizes}\define{majorize} another \(\sum c_a x^a\) if \(b_a \ge \left|c_a\right|\) for all \(a\).
If a convergent series majorizes another, the other is absolutely convergent.
If \(f\) majorizes \(g\) and \(u\) majorizes \(v\) then \(f \circ u, f+u, fu, \partial^a \! f\) majorizes \(g \circ v, g+v, gv, \partial^a \! g\) respectively. 

The geometric series
\[
 \frac{1}{1-t}=1+t+t^2+\dots
\]
has a Taylor series with positive coefficients.
A formal Taylor series \(c_0+c_1 t + \dots\) with all \(\left|c_j\right|<1\) converges absolutely for \(|t|<1\), because the geometric series converges and majorizes it.
Rescaling \(t\) and our series, we see that if a formal Taylor series is majorized by a geometric series, then it converges near the origin, i.e. if \(\left|c_j\right|\) are bounded by a geometric series, in other words \(\left|c_j\right| \le Mr^j\) for some \(M, r>0\), then \(c_0 + c_1 t + \dots\) converges absolutely near the origin.
An \emph{analytic function}\define{analytic function} is one which is locally the sum of a convergent Taylor series.

\begin{lemma}\label{lemma:cauchy:convergent.Taylor}
A formal Taylor series converges to an analytic function just when it is majorized by a product of geometric series in one variable each.
\end{lemma}
\begin{proof}
We give the proof for one variable around the point \(x=0\), and let the reader generalize.
Take any analytic function \(f(x)\) with convergent Taylor series \(f(x)=\sum c_n x^n\).
Since it converges absolutely for \(x\) near \(0\), \(\sum \left|c_n\right| r^n\) converges for \(r\) near \(0\), so the terms of this series are bounded.
Rescale to get a bound of \(1\), i.e. \(\left|c_n\right| r^n < 1\) for all \(n\).
Therefore 
\[
\left|c_n\right| \le \frac{1}{r^n}
\]
i.e. \(f(x)\) is majorized by \(1/(1-x)\).
\end{proof}
\prob{cauchy-kov:u.con}{Prove that any two formal Taylor series which converge near the origin to the same analytic function agree.}
\begin{answer}{cauchy-kov:u.con}
\cite{Serre:2006}, p. 68. 
\end{answer}
\prob{cauchy-kov:u.pwr.an}{Prove that an analytic function given by a convergent Taylor series around some point is also given by a convergent Taylor series around every nearby point.}
\begin{answer}{cauchy-kov:u.pwr.an}
\cite{Serre:2006}, p. 69. 
\end{answer}
\prob{cauchy-kov:u.a.c}{Prove that any two analytic functions defined on a connected open set which agree near some point agree everywhere.}
The derivative and composition of analytic functions is analytic, and the analytic inverse and implicit function theorems hold with the usual proofs \cite{Krantz/Parks:2002,Serre:2006}.

\section{Complexification}
Every Taylor series converging in some domain of real variables continues to converge for complex values of those variables, with small imaginary parts, by the same majorization.
Hence every analytic function of real variables extends to an open set in a domain of complex variables.
By elementary complex analysis \cite{Ahlfors:1978}, sums, differences, products and derivatives of analytic functions on an open set are analytic, as are quotients, wherever the denominator doesn't vanish, and the analytic inverse and implicit function theorems hold.


\chapter{Proof of the Cartan--K\"ahler theorem}
\todo{We have a better version of the following theorem already proven.}
\section{Restating the Cartan--K\"ahler theorem}
\begin{theorem}[Cartan--K\"ahler \Romanbar{2}]\define{Cartan--K\"ahler theorem}\define{theorem!Cartan--K\"ahler}
Take an analytic exterior differential system on a manifold \(M\).
Take an analytic embedded integral submanifold \(X\), say of dimension \(p\), so that its tangent spaces are regular.
Take an analytic embedded submanifold \(R\) containing \(X\) so that, for each \(x \in X\), the polar equations of \(T_x X\) pull back by linear isomorphism to \(T_x R\), and the vectors in \(T_x R\) satisfying the polar equations of \(T_x X\) form a \((p+1)\)-plane.
Then \(X\) is a hypersurface in a locally unique analytic integral submanifold.
\end{theorem}
\begin{proof}
Restrict \(\II\) to \(R\) and apply theorem~\vref{theorem:CKI}.
\end{proof}
The Cartan--K\"ahler theorem, as stated in theorem~\vref{theorem:CK}, follows by induction starting with \(X\) a point.

\section{The noncharacteristic initial value problem}
\todo{I want this kind of thing, but I might be able to use a less arduous statement, without regularity.}
\begin{theorem}[Cartan--K\"ahler I]%
\label{theorem:CKI}%
\define{Cartan--K\"ahler theorem}%
\define{theorem!Cartan--K\"ahler}
Take an analytic exterior differential system.
Every regular analytic noncharacteristic integral submanifold is a hypersurface in an analytic integral submanifold.
Any two such submanifolds contain identical open subsets containing the common hypersurface.
\end{theorem}
\begin{proof}
Take a regular noncharacteristic integral manifold \(X\) and one of its tangent spaces \(E\defeq T_x X\).
Let \(E_+\) be the unique integral element in which \(E\) is a hyperplane.
We can assume that \(X\) is a \((p-1)\)-dimensional manifold.
We can assume that \(E_+\) is a maximal integral element.
Since \(E\) is noncharacteristic, \(s_p=0\), i.e. there are no free derivatives in grade \(p\).
Therefore there are no \(\omega^p\) factors in the \(\omega^I\) in any free derivative
\[
\vartheta = \dots + \freeDeriv{\pi^{\alpha}} \wedge \omega^I + \dots
\]
in our tableau.
Wedge up \(\vartheta\) with various \(\omega^i\) until it becomes a \(p\)-form.
There is exactly one choice of various \(\omega^i\) to wedge with so that we get a \(p\)-form, call it \(\vartheta^{\alpha}\), of the form 
\[
\vartheta^{\alpha} = \dots + \freeDeriv{\pi^{\alpha}} \wedge \omega^{1\dots p-1} + \dots
\]
If we make a different choice of \(\omega^i\) to wedge up, we end up with some \(p\)-form
\[
\vartheta^{\mu} = \dots + \freeDeriv{\pi^{\alpha}} \wedge \omega^{1\dots \hat\imath \dots p} + \dots
\]
In particular, we must wedge with \(\omega^p\) to make this \(\vartheta^{\mu}\), since we started in grade less than \(p\).
The equation \(0=\vartheta^{\alpha}\), written in coefficients of \(p\)-dimensional integral elements, is some equation \(p^{\alpha}_p=\dots\) for the highest of the high coefficients, which we write out in terms of low coefficients.
The equation \(0=\vartheta^{\mu}\) is \(p^{\alpha}_i=\dots\) for the other, not quite so high, high coefficients, in terms of low coefficients.
By involutivity, these equations suffice to determine the integral elements, so we can assume that \(\II^p\) contains only these without loss of generality.

Write \(\bar\vartheta\) to mean \(\vartheta\) in the linearization, i.e. after we drop \(\pi\wedge\pi\) terms.
The symbol of \(\II\) is 
\[
\xi \wedge 
\begin{pmatrix}
e_i \hook \bar\vartheta^{\alpha}\\
e_i\hook \bar\vartheta^{\mu}
\end{pmatrix}.
\]
The hyperplane \(E\subset E_+\) is not characteristic, i.e. the point \(E \in \mathbb{P}E^*\) does not belong to \(\CV_{E_+}\), i.e. if we set \(\xi=\omega^p\), the symbol is not zero.
But the final rows are multiples of \(\omega^p\), so at \(E\), the symbol of \(\II\) is 
\[
0\ne
\omega^p \wedge 
\begin{pmatrix}
e_i \hook \bar\vartheta^{\alpha}\\
0
\end{pmatrix}.
\]

Let \(\JJ\subset \II\) be the exterior differential system generated by all of the \(p\)-forms \(\vartheta^{\alpha}\), and put also all forms of higher degree into \(\JJ\).
The character \(s_{p-1}\) of \(\JJ\) is the number of polar equations of \(\II\), i.e. \(s_0+s_1+\dots+s_{p-1}\) of \(\II\).
We can assume that both \(\II\) and \(\JJ\) have maximal integral elements of dimension \(p\), \(\JJ\) has \(p+s_{p-1}=\dim M\).
In particular, \(\sigma(\xi)\ne 0\) at \(\xi=\omega^p\), i.e. \(E\) is not characteristic for \(\JJ\).
Hence the system associated to \(\JJ\) is determined and noncharacteristic at \(E\).
A \(p\)-dimensional \(\JJ\)-integral manifold \(X_+\) containing \(X\) is guaranteed by the Cauchy--Kovalevskaya theorem.
But is \(\II=0\) on \(X_+\)?

On \(X_+\), by definition, the \(p\)-forms \(\vartheta^{\alpha}\) vanish.
So \(X_+\) is an \(\II\)-integral manifold if and only if the \(p\)-forms \(\vartheta^{\mu}\) vanish on \(X_+\).
The \(\omega^i\) coframe \(X_+\), while \(0=\theta^a\), and all free derivatives \(\pi^{\alpha}\) are now multiples of \(\omega^i\), i.e. become torsion.
So on \(X_+\), all characters of \(\II\) are zero.
So on \(X_+\), we have only torsion terms.
We apply the following lemma to \(\II\) on \(X_+\).
\end{proof}

\todo{Got rid of that lemma.}

\section{Example: triply orthogonal webs, noncharacteristic data}
\begin{theorem}
Take an analytic Riemannian 3-manifold \(X\) with orthonormal frame bundle \(M\).
Take an embedded analytic surface \(Y \subset X\) and pick an orthonormal basis \(e_1,e_2,e_3\) of \(T_x X\) at each point \(x \in Y\), so that none of \(e_1,e_2,e_3\) is tangent to \(Y\).
Then there is a unique triply orthogonal web on \(X\) near \(Y\) consisting of three foliations whose leaves are perpendicular to \(e_1,e_2,e_3\) respectively at each point of \(Y\).
\end{theorem}
\begin{proof}
Recall that each integral element is uniquely determined by its equation
\[
\gamma=
\begin{pmatrix}
0 & p_{12} & p_{13} \\
p_{21} & 0 & p_{23} \\
p_{31} & p_{32} & 0 
\end{pmatrix}
\omega,
\]
where the numbers \(p_{ij}\) can be arbitrarily chosen.
For example, if we take all of these \(p_{ij}\) in the first two columns to be zero, we find that the plane \((\omega_3=0)\) lies in all of those integral elements, so lies in the characteristic variety of all of those integral elements.
More generally, two such sets of equations \(\gamma=p \omega\), \(\gamma=(p+q)\omega\) agree on a \(2\)-plane just when their difference vanishes on a \(2\)-plane, i.e. just when \(q=0\) on some \(2\)-plane, i.e. just when \(q\) has rank 1:
\[
q=
\begin{pmatrix}
0 & q_{12} & q_{13} \\
q_{21} & 0 & q_{23} \\
q_{31} & q_{32} & 0 
\end{pmatrix}.
\]
\prob{rank.1}{Linear algebra: prove that \(q\) has rank \(1\) just when either \(q=0\) or \(q\) vanishes on a \(2\)-plane which contains one of the coordinate axes.}
\begin{answer}{rank.1}%
Suppose that \(q\) has rank \(1\):
\[
q=
\begin{pmatrix}
0 & q_{12} & q_{13} \\
q_{21} & 0 & q_{23} \\
q_{31} & q_{32} & 0 
\end{pmatrix}.
\]
Every row is a multiple of any nonzero row, and the same for every column.
At least one entry must nonzero; permute indices to get \(q_{12}\ne 0\).
The zero in \(q_{11}\) ensures that column \(1\) is zero, and the zero in \(q_{22}\) ensures that row \(2\) is zero.
By the same reasoning, if in addition \(q_{13}\ne 0\), then the zero in \(q_{33}\) ensures that row \(3\) is zero:
\[
q=
\begin{pmatrix}
0 & q_{12} & q_{13} \\
0 & 0 & 0 \\
0 & 0 & 0 
\end{pmatrix}
\]
So \(q=0\) precisely on the \(2\)-plane \(0=q_{12}y+q_{13}z\), containing the \(x\)-axis.
On the other hand, if \(q_{13}=0\),
\[
q=
\begin{pmatrix}
0 & q_{12} & 0 \\
0 & 0 & 0 \\
0 & q_{32} & 0 
\end{pmatrix}.
\]
So \(q=0\) precisely on \(y=0\).
Similarly if we permute indices.
\end{answer}
So the characteristic variety consists precisely of the hyperplanes \(0=\xi_1\omega_1+\xi_2\omega_3+\xi_3\omega_3\) which contain one of the coordinate axes, i.e. where \(\xi_1=0\) or \(\xi_2=0\) or \(\xi_3=0\), i.e. hyperplanes in which one of \(e_1,e_2,e_3\) are tangent to the initial data surface \(Y\).
\end{proof}


